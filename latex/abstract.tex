\thispagestyle{empty}

\tocLessLeft{Streszczenie pracy}
Praca przedstawia projekt systemu sterowniczego do różnorodnych silników elektrycznych, na który składają się między innymi: Zbudowane w oparciu o 32 bitowy mikrokontroler urządzenie do którego podłączany jest silnik i peryferia (sterownik) oraz aplikacja PC-owa służąca do nastaw niezbędnych do poprawnego działania układu parametrów brzegowych oraz badania pracy zasilanego przez sterownik silnika. Rozwiązanie ma zostać zastosowane do sterowania elementami ruchomymi w maszynach typu {\it Pick and Place}, drukarkach szablonowych, drukarkach 3D oraz we wszelkiego rodzaju konwojerach. W części opisowej dokumentu, została zaprezentowana niezbędna teoria nt. silników elektrycznych oraz sterowania. Ponadto, zostały porównane komercyjne rozwiązania sterowników różnej klasy, które są dostępne na rynku.

\tocLessLeft{Słowa kluczowe}
Sterowanie, silnik, elektronika mocy, oprogramowanie

\tocLessLeft{Abstract}
The paper presents a project of a control system, that can be used with a various types of electric motors. Among others the main components of the system are: built upon a 32 bit microcontroller device, to which the motors and peripherals are connected (the driver) and a PC application used for defining critical operating parameters and for gathering connected motor work diagnostics . This application is to be used with Pick and Place machines, stencil printers, 3D printers and all sort of conveyors. In the generic part of the text, the essential theory of electric motors and control has been described. Also, commercially available on the market  motor drivers of various class has been compared.

\tocLessLeft{Key words}
Controll, motor, power electronics, software


\clearpage
