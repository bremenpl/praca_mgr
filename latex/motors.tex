\section{Podstawowe informacje o silnikach}

Rozdział ten przedstawia podział maszyn elektrycznych ze względu na ich różne parametry, oraz opis tych silników, które są (i mogą być) wspierane przez skonstruowany sterownik. 
Rozdział został opracowany głównie na podstawie źródeł \cite{przepiorkowski} i \cite{jones}.

\subsection{Podział silników elektrycznych}

Istnieje bardzo wiele rodzajów silników, a ponadto cały czas powstają nowe konstrukcje, które trudno jest sklasyfikować przy pomocy klasycznych definicji. Podział na silniki AC i DC nie wystarczy-- w przypadku zastosowania elektronicznego modułu sterującego rodzaj zasilania silnika nie ma tak dużego znaczenia. Nie można także jednoznacznie stwierdzić, czy maszyna z wirującymi magnesami to silnik prądu stałego, czy przemiennego. Biorąc to pod uwagę, silniki elektryczne można podzielić na trzy podstawowe grupy.

\tocLessLeftNorm{Silniki prądu stałego -- DC} 

Zaliczane są tu silniki komutatorowe:
\begin{easylist}
	& z magnesem trwałym (PMDC),
	& szeregowe,
	& bocznikowe,
	& szeregowo--bocznikowe.
\end{easylist} 

\tocLessLeftNorm{Silniki prądu przemiennego -- AC}

Do tej grupy  zaliczane są silniki zasilane bezpośrednio z sieci prądu przemiennego. W tym przypadku podział jest bardziej skomplikowany, ze względu na większą ilość kryteriów:
\begin{easylist}
	& \underline{Silniki asynchroniczne} -- charakteryzuje je tzw. ,,poślizg''. Rotor takiego silnika wiruje z niższą prędkością od prędkości wirowania pola magnetycznego. Najpopularniejszymi silnikami z tej grupy są silniki klatkowe:
	
		&& 1--fazowe kondensatorowe, 
		&& 1--fazowe ze zwartą fazą rozruchową, 
		&& 1--fazowe z odłączanym uzwojeniem rozruchowym, 
		&& 1--fazowe trójfazowe, 
	
	& \underline{Silniki synchroniczne} -- rotor wiruje z prędkością równą prędkości wirowania pola magnetycznego. Mogą być jednofazowe lub trójfazowe. Ze względu na konstrukcję, silniki te dzielą się na reluktancyjne lub z wirującym magnesem.
	& \underline{Silniki komutatorowe uniwersalne} -- komutatorowe silniki szeregowe przystosowane konstrukcyjnie do zasilania z jednej fazy prądu przemiennego.
\end{easylist} 

\tocLessLeftNorm{Silniki z komutacją elektroniczną}

Do tej grupy zalicza się silniki, których działanie nie było by możliwe bez elektronicznego układu sterującego (sterowanie impulsowe). Najpopularniejsze z nich to:
\begin{easylist}
	& \underline{Silniki krokowe} -- ze względu na konstrukcję wyróżnia się silniki:
		&& o zmiennej reluktancji,
		&& z magnesem trwałym,
		&& hybrydowe.
\end{easylist} 
Ze względu na sposób sterowania uzwojeniami:
\begin{easylist}
		&& silniki unipolarne,
		&& silniki bipolarne.
\end{easylist} 
	Daje to sześć różnych kombinacji konstrukcyjnych silników krokowych.
	
\insertImg{grafiki/stepper_nema17.png}
	{Przykładowy silnik krokowy hybrydowy ST4118 w standardzie Nema17, produkowany przez firmę Nanotec}
	{nanotec}	
	
Nema ({\em National Electrical Manufacturers Association}) to stowarzyszenie producentów urządzeń elektrycznych, którego siedziba mieści się w USA. Standard Nema określa między innymi wymiary silników krokowych. Numer 17 oznacza szerokość i wysokość silnika 1.7 x 1.7 cala (43.2 x 43.2 mm).

\insertImg{grafiki/bldc_db42.png}
	{Przykładowy silnik BLDC (model DB42), produkowany przez firmę Nanotec}
	{nanotec}

\begin{easylist}
	& \underline{Silniki bezszczotkowe z wirującym magnesem (BLDC)} -- w literaturze angielskiej silniki te są opisywane jako prądu stałego ({\em BrushLess DC}). W przypadku silników dwubiegunowych z czujnikami Halla (np. tego typu, które są używane w małych wentylatorach do chłodzenia radiatorów) taki opis może być uzasadniony. Nie można jednak jednoznacznie zaliczyć do grupy DC silników wielobiegunowych trójfazowych z wirującym magnesem \cite{przepiorkowski}. Mylne może być także to, że w przypadku silników z wbudowanym sterownikiem nie różnią się one z punktu użytkownika od silnika komutatorowego z magnesem trwałym. Maszyny BLDC charakteryzują się bardzo wysoką prędkością obrotową, w szczególności w porównaniu do silników krokowych.	
	
	& \underline{Silniki liniowe} -- wynikiem działania takiej maszyny jest przemieszczenie liniowe, a nie ruch obrotowy. Konstrukcja takiego silnika w dużym uproszczeniu polega na ,,rozwinięciu'' wirnika i stojana na płaszczyźnie. Nazwa silnik liniowy lub {\em Linear Actuator} bardzo często dotyczy zwykłego silnika krokowego z wbudowaną przekładnią śrubową zamieniającą ruch obrotowy na posuwisto--zwrotny, co bardzo często wprowadza niejasności i wątpliwości podczas projektowania systemów napędowych.
	
	\insertImg{grafiki/linear_actuator_l35.png}
		{Przykładowy silnik krokowy hybrydowy L35 z przekładnią śrubową, produkowany przez firmę Nanotec}
		{nanotec}
	
	& \underline{Silniki liniowe VCM} -- zasada działania silnika VCM ({\em Voice Coil Motor}) lub VCA ({\em Voice Coil Actuator}) jest bardzo zbliżona do zasady działania cewki głośnikowej w szczelinie magnesu. Przesunięcie liniowe tych silników nie przekracza najczęściej 50 mm.		
	
\end{easylist} 

\insertImg{grafiki/vcm_moticont.png}
	{Przykładowy silnik liniowy VCM (model GVCM-051-051-01), produkowany przez firmę Moticont}
	{moticont}
	
\subsection{Parametry silników}

Podstawowe parametry silnika są czasami podawane na jego tablicy znamionowej. Pozostałe należy odszukać w jego karcie katalogowej lub oszacować według właściwości poszczególnych rodzajów silników.
\\

\begin{easylist}

	& Moc znamionowa: podawana jest moc elektryczna w watach pobierana przez silnik przy pracy z normalną prędkością obrotową. Moc mechaniczna jest mniejsza i zależy od sprawności silnika (typowo od 40 do 80 \%).
	\\
	& Napięcie zasilania: znamionowa wartość napięcia zasilającego (stałego lub zmiennego), przy której określane są pozostałe parametry. Najczęściej silniki mogą być zasilane napięciem niższym. Nie należy stosować napięć dużo wyższych od znamionowego (max. +15 \%).
	\\
	& Moment obrotowy: parametr ten jest podawany w Niutono--metrach $ [Nm] $. Typowe wartości dla małych silników: od 0.01 do 50 Nm. Dla większości rodzajów silników występuje zależność momentu obrotowego i natężenia prądu pobieranego przez maszynę.
	\\
	& Moment rozruchowy: ważny parametr informujący o tym czy silnik jest w stanie wystartować pod obciążeniem. Podawany w Nm lub w procentach momentu obrotowego. W zależności od rodzaju silnika moment może być mały (do 150 \%), średni (150 - 250 \%) lub duży (> 250 \%).
	\\
	& Obroty znamionowe: wyrażane w RPM-ach ({\em Revolutions per minute}. Zawsze podawane są przy znamionowym obciążeniu i napięciu zasilania, lub podana jest charakterystyka obrotów w funkcji obciążenia. Obroty silników zawierają się w granicach od 100 do 100000 RPM, ale najczęściej spotykane wartości to od 1000 do 4000 RPM. Dla maszyn prądu przemiennego podawane są obroty przy określonej częstotliwości napięcia sieciowego (50 lub 60 Hz). Dla silników krokowych obroty znamionowe nie są podawane. Zamiast tego występują obroty maksymalne lub dopuszczalna częstotliwość impulsów.
	\\
	& Prąd znamionowy i rozruchowy: pobór prądu w normalnych warunkach pracy i w czasie rozpędzania silnika. Ten drugi może być nawet 2 - 8 krotnie większy od znamionowego (największe prądy rozruchowe mają silniki indukcyjne). Prąd rozruchowy nie występuje w silnikach sterowanych impulsowo.
	\\
	& Kierunek obrotów: parametr dotyczący przede wszystkim silników komutatorowych szeregowych i niektórych silników indukcyjnych jedno fazowych. Dla tych silników kierunek obrotów jest wymuszony przez konstrukcję silnika. \newline 
	W pozostałych maszynach kierunek zależy od polaryzacji przyłożonego do ich zacisków napięcia lub w przypadku maszyn ze sterowaniem impulsowym-- od kolejności podawanie impulsów na poszczególne uzwojenia.
	\\
	& Konstrukcja mechaniczna: wymiary, ciężar, średnica wału, mocowania, rodzaj łożysk wirnika itp.
	\\
	& Konstrukcja obudowy, chłodzenie: spotykane są silniki o konstrukcji otwartej, w której uzwojenia są dostępne z zewnątrz lub zamkniętej w której silnik jest całkowicie obudowany. W silnikach otwartych często wymagane jest chłodzenie strumieniem powietrza-- np. silniki odkurzaczy, czy wiertarek.  

\end{easylist} 

\subsection{Silnik liniowy (VCM)}

Pierwszym z silników obsługiwanych przez projektowany sterownik jest silnik liniowy VCM. Jego konstrukcja jest bardzo prosta, aczkolwiek sam proces technologiczny wymagany do jego wytworzenia już nie. Z tego właśnie powodu VCM nawet bardzo małej mocy są drogie (Cena od \$100 za pojedyncze sztuki). Ciężko jest tak naprawdę stwierdzić czy jest to silnik czy siłownik, bo zakres jego ruchu nie przekracza zwykle kilku centymetrów.

\insertImg{grafiki/vcm_budowa.png}
	{Budowa cylindrycznego silnika VCM}
	{przepiorkowski}

Nazwa silnika pochodzi od jego zasady działania- identycznie jak w zwykłym głośniku jest to cewka poruszająca się w szczelinie magnesu. Zgodnie z regułą Lorentza:

\begin{defn}
	Jeżeli przez cewkę znajdującą się w polu magnetycznym przepływa prąd, to na cewkę działa siła proporcjonalna do natężenia prądu, a napięcie samoindukcji jest proporcjonalne do szybkości poruszania się cewki. \cite{przepiorkowski}.
\end{defn}

Podstawowe części składowe silnika VCM to:
\begin{easylist}
	& dwa magnesy trwałe ułożone tak, aby były skierowane tym samym biegunem w stronę cewki,
	& cewka poruszająca się w szczelinie między magnesami,
	& obudowa z miękkiego żelaza zamykająca obwód magnetyczny.
	\\
\end{easylist}

Zmieniając natężenie i polaryzację przepływającego prądu, możliwe jest bardzo precyzyjne sterowanie położeniem cewki. Ze względu na to układ elektroniczny sterujący silnikiem nie wymaga wielu komponentów, a algorytm sterowania nie jest skomplikowany to VCM ma jednak jeszcze jedną zaletę, która sprawia że jest on wybierany do niektórych aplikacji zamiast np. tańszego silnika krokowego z przekładnią śrubową- pozwala wykonywać bardzo dynamiczne ruchy i posiada wysoką zwrotność. ,,Liniowy'' silnik krokowy o podobnej mocy nawet z zastosowaniem śruby o dużym skoku, nie będzie w stanie osiągnąć podobnych prędkości przy zachowaniu korzystnego momentu. Brak konieczności zamiany ruchu obrotowego na posuwisto-zwrotny eliminuje błędy pozycji (które mogą wynikać np. z luzów w przekładni). Silniki VCM stosowane są wszędzie tam, gdzie potrzebne jest dokładne i szybkie ustawienie pozycji, np. w precyzyjnych urządzeniach mechanicznych i optycznych. \\

Aby efektywnie i bezpiecznie korzystać z silników VCM w danej aplikacji, należy mieć na uwadze kilka ważnych cech tych urządzeń:

\begin{easylist}
	& Korpus cewki nie jest w żaden sposób przymocowany do obudowy silnika. Oznacza to że to projektant musi zadbać o to, aby korpus wsuwał i wysuwał się z cylindra równolegle, stosując odpowiednie mocowanie mechaniczne obu części. Nie zapewniwszy tego, uzwojenie silnika jest narażone na starcie się lakieru izolacyjnego, poprzez ocieranie się o obudowę. 
	
	& W zależności od aktualnego położenia (wysunięcia) korpusu, zależność między siłą a przepływającym przez uzwojenie prądem jest zmienna.
	
	\insertImgSetSize{grafiki/vcm_force_current_plot.eps}
		{80}
		{Przykładowy przebieg siły $ [\frac{N}{A}] $ pchającej/ ciągnącej w funkcji wysunięcia korpusu dla silnika liniowego VCM 019-048-02, firmy Moticont.}
		{moticont}
	
Graf \ref{grafiki/vcm_force_current_plot.eps} prezentuje zależność między siłą silnika a położeniem jego cewki. Jak widać największa i stała siła ($ 2 \frac{N}{A} $) występuje dopiero kiedy korpus jest trochę wysunięty, i zaczyna lawinowo spadać przy końcowym wysuwie. Jeżeli układ mechaniczny, którego częścią jest silnik, nie zostanie wyposażony w odpowiednią blokadę, korpus może wypaść z cylindra ponieważ silnik nie będzie już w stanie go utrzymać.

	& Producent w karcie katalogowej silnika podaje parametr maksymalnej ciągłej mocy ({\em Max continuous power}) i odpowiada ona zazwyczaj prądowi, który płynie przez cewkę w momencie niepełnego wysunięcia (należy oczywiście wziąć pod uwagę fizyczne obciążenie części ruchomej- masę obudowy lub korpusu, w zależności od tego która z części jest przymocowana, a która jest w ruchu). Przy dużym wysunięciu zależność siły do płynącego przez cewkę prądu drastycznie spada i utrzymując korpus długo w takiej pozycji można w bardzo łatwy sposób spalić uzwojenie silnika. \\
	
\end{easylist}

Przy zastosowaniu odpowiedniej konstrukcji układu napędowego (zachowana równoległość części mechanicznych) i sterowania zapewniającego ochronę przed zbyt dużym prądem płynącym przez cewkę, silnik VCM z czasem nie ulega właściwie żadnej degradacji i długość jego działania jest ograniczona żywotnością części mechanicznych układu.










\subsection{Silnik krokowy}

Silniki krokowe są maszynami elektrycznymi bez komutatorowymi. Zazwyczaj wszystkie uzwojenia silnika są częścią stojana, podczas gdy rotor jest magnesem trwałym lub w przypadku silników VRM ({\em Variable Reluctance Motor}) zębatym blokiem z materiału magnetycznie miękkiego (patrz sekcja \ref{subsec:materialy_fer}). Komutacja (czyli zmiana kierunku/ uzwojenia w którym następuje przepływ prądu elektrycznego) musi być wymuszona poprzez zewnętrzne urządzenie sterujące maszyną- takie sterowniki są zazwyczaj projektowane w taki sposób, aby zapewnić możliwość utrzymania wirnika w niemal dowolnej pozycji kątowej, oraz obracać nim w obu kierunkach. 

\subsubsection{Rodzaje silników krokowych}

Silniki krokowe ze względu na konstrukcję można podzielić na trzy główne grupy: 
\begin{easylist}
	& silniki o zmiennej reluktancji,
	& z magnesem trwałym,
	& hybrydowe.
\end{easylist}

\insertImg{grafiki/reluctance_moment.png}
		{Powstawanie reluktancyjnego momentu obrotowego.}
		{przepiorkowski}

Reluktancja jest parametrem analogicznym do rezystancji elektrycznej, lecz odniesionym do strumienia magnetycznego- jest to inaczej ,,rezystancja magnetyczna'' (patrz sekcja \ref{subsec:reluktancja}). Podobnie jak prąd płynie drogą o najmniejszej rezystancji, tak linie sił pola skupiają się w obszarze o najmniejszej reluktancji. Rysunek \ref{grafiki/reluctance_moment.png} przedstawia powstawanie reluktancyjnego momentu obrotowego. Strumień indukcji $ \Phi $ to funkcja prądu uzwojenia $ I $ i reluktancji obwodu magnetycznego $ Rm $. 

\begin{equation} \label{eq:steper1}
	\Phi = \frac{I}{Rm}
\end{equation}

Obrócenie ruchomego elementu o kąt $ \alpha $ spowoduje że będzie on próbował powrócić do położenia $ \alpha = 0 $, w którym reluktancja obwodu jest najmniejsza. W silnikach VRM przepływ prądu stałego przez uzwojenia powoduje, że zęby wirnika ustawiają się naprzeciw zasilanego uzwojenia.

\insertImg{grafiki/vrm_przekroj.png}
		{Silnik o zmiennej reluktancji (VRM)- Przekrój.}
		{przepiorkowski}

W odróżnieniu do VRM, wirnik silnika krokowego PM ({\em Permanent Magnet}- Magnes trwały) nie posiada zębów. Jest wykonany w postaci walca naprzemiennie namagnesowanego biegunami N i S. Specyficzny rodzaj zębów ma rdzeń stojana. W zależności od kierunku przepływu prądu w uzwojeniu przyciągane są odpowiednie bieguny wirnika (resunek \ref{grafiki/pm_stepper.png}).

\insertImg{grafiki/pm_stepper.png}
		{Zasada działania silnika PM.}
		{ni}
		
Główną zaletą silnika krokowego PM jest zastosowanie magnesów trwałych w stojanie, dzięki czemu nie ma potrzeby stosowanie szczotek jak w silnikach DC oraz jego niska cena. Wadą tego typu maszyny jest relatywnie niski moment obrotowy i brak możliwości rozwijania dużych prędkości obrotowych. \\

Silnik hybrydowy łączy w sobie cechy obu rozwiązań, dzięki czemu zostały poprawione takie parametry jak:
\begin{easylist}
	& moment obrotowy, 
	& maksymalna prędkość obrotowa, 
	& rozdzielczość kroku.
	\\
\end{easylist}

Niestety silniki hybrydowe są około 2-3 razy droższe od silników PM. Wirnik silnika HB ({\em Hybrid Motor}) jest zbudowany z uzębionych nabiegunników i magnesu trwałego, powodującego naprzemienne magnesowanie zębów biegunami N i S. Uzębiony stojan konstrukcją przypomina ten z silnika VRM. 

\insertImgSetSize{grafiki/stepper_budowa.png}
		{70}
		{Konstrukcja dwufazowego silnika krokowego hybrydowego.}
		{enggarage}
		
W silniku HB wirujące pole stojana ,,przerzuca'' wirnik z jednego położenia do drugiego na zasadzie jak w silniku VRM. Jest to możliwe dzięki przesunięciu ,,północnej'' i ,,południowej'' części wirnika o pół ząbka. Silniki HB dzięki poprawionym parametrom są obecnie najbardziej popularnymi silnikami krokowymi, pomimo niskiej ceny silników PM.
		
\insertImgSetSize{grafiki/stepper_rotor.png}
		{70}
		{Budowa wirnika silnika krokowego hybrydowego dwufazowego.}
		{edw_sierpien_2002}

\subsubsection{Sterowanie}

W pierwszej kolejności należy rozróżnić sposób zasilania silnika krokowego, który może być unipolarny lub bipolarny (rys. \ref{grafiki/stepper_zasilanie_uni_bi.png}).

\insertImg{grafiki/stepper_zasilanie_uni_bi.png}
		{Sposób sterowania (zasilania) silnika krokowego dwufazowego- z lewej unipolarnie, z prawej bipolarnie.}
		{hackaday}

Przy zasilaniu unipolarnym, dla pojedynczego uzwojenia na zewnątrz silnika wyprowadzone są trzy przewody- Oba końce cewki i odczep w połowie jej długości. Podłączając do odczepu ,,+'' zasilania można sterować połówkami uzwojeń zwierając w odpowiedniej kolejności i na odpowiedni czas końcówki do masy. W zasilaniu bipolarnym, prąd płynie zawsze przez całe uzwojenie (z silnika wyprowadzone są tylko dwa przewody na cewkę). Zaletą wariantu unipolarnego jest możliwość znacznego uproszczenia układu sterowania. Poważną wadą natomiast jest to, że prąd podawany jest zawsze tylko na połowę danego uzwojenia, co ma negatywne przełożenie na moment obrotowy. 

\insertImg{grafiki/stepper_simple.png}
		{Uproszczony model silnika krokowego dwufazowego o kroku 90\degree.}
		{howtomecha}

Oddzielną sprawą jest sposób/ sekwencja podawania impulsów prądowych do uzwojeń silnika. Sterowanie pod tym względem dzieli się na:

\begin{easylist}
	& Falowe, 
	& Pełnokrokowe, 
	& Półkrokowe,
	& Mikrokrokowe.
	\\
\end{easylist}

Do pomocy w wyjaśnieniu poszczególnych typów sterowania posłuży rysunek \ref{grafiki/stepper_simple.png}, prezentujący uproszczony silnik krokowy. Maszyna składa się z wirnika zbudowanego jak na rys. \ref{grafiki/stepper_rotor.png} tyle że z ilością zębów (na wirniku i stojanie) pozwalającą osiągnąć pełen obrót w czterech pełnych komutacjach. Silnik posiada 2 uzwojenia A i B nawinięte w taki sposób, że pary A/A' i B/B' są ze sobą połączone (uzwojenie podzielone na dwie sekcje). Przykładając napięcie do cewki A, w zależności od polaryzacji tego napięcia, prąd będzie wpływał przez uzwojenie A i wypływał z uzwojenia A' lub na odwrót. Tak samo ma się sytuacja w przypadku cewki B.

\tocLessLeftNorm{Sterowanie falowe}

W przypadku braku zasilania, wirnik silnika ustawi się w jednej z czterech spoczynkowych pozycji (0\degree, 90\degree, 180\degree lub 270\degree). Dzieje się tak dlatego, że namagnesowany wirnik usiłuje zająć taką pozycję, aby reluktancja obwodu magnetycznego była jak najmniejsza, tzn. aby namagnesowane zęby wirnika były ustawione w jednej linii z dowolną parą zębów stojana. Znikomy moment obrotowy powodujący takie pozycjonowanie to inaczej moment bezprądowy (spoczynkowy). 

\insertImgSetSize{grafiki/stepper_sterowanie_falowe.eps}
		{100}
		{Sekwencja zasilania uzwojeń przy sterowaniu falowym (obrót wirnika z rys. \ref{grafiki/stepper_simple.png} według wskazówek zegara).}
		{oprWlasne}

Rysunek \ref{grafiki/stepper_sterowanie_falowe.eps} prezentuje sekwencję dzięki której wirnik obróci się o trzy pełne obroty w prawo przy zastosowaniu sterowania falowego. Na osi odciętych znajdują się kolejne chwile w czasie, a na osi rzędnych wartość potencjału przyłożonego do danego zacisku (wartości 1 i 0 prezentujące stan wysoki lub niski potencjału). Aby wirnik obracał się w przeciwnym kierunku, sekwencję należy odtworzyć w odwrotnej kolejności. Jak widać na przebiegach, w danym momencie w czasie prąd znajduje się zawsze tylko w jednym z uzwojeń, co nie umożliwia uzyskania maksymalnego dla danej maszyny momentu obrotowego. Pomimo że sterowanie falowe nie zapewnia uzyskania optymalnego momentu obrotowego, to przy zastosowaniu w niektórych aplikacjach jest wystarczające, a ponadto jest jest najprostsze do zaimplementowania.

\tocLessLeftNorm{Sterowanie pełnokrokowe}

W trybie sterowania pełnokrokowego prąd płynie przez oba uzwojenia jednocześnie. Nie ma to wpływu na prędkość obrotową wirnika, ale ma znaczący wpływ na moment obrotowy, który jest dwa razy większy niż w trybie falowym. Pełny obrót wirnika dalej jest osiągany w czterech "ruchach", ale absolutna pozycja kątowa w każdym z nich jest względem trybu falowego przesunięta o 45\degree.

\insertImgSetSize{grafiki/stepper_sterowanie_krokowe.eps}
		{100}
		{Sekwencja zasilania uzwojeń przy sterowaniu pełnokrokowym (obrót wirnika z rys. \ref{grafiki/stepper_simple.png} według wskazówek zegara).}
		{oprWlasne}
		
Rysunek \ref{grafiki/stepper_sterowanie_krokowe.eps} przedstawia przebiegi dla 3 obrotów, sterując pełnokrokowo. Kolejne pozycje wirnika są następujące: 45\degree w pierwszym interwale czasowym, 135\degree w drugim, 225\degree w kolejnym i 315\degree w ostatnim (dla danego obrotu). Potem cykl się powtarza. Tryb pełnokrokowy pozwala osiągnąć optymalny dla danego silnika moment obrotowy, ale niestety rozdzielczość obrotu jest ograniczona do ilości kroków na obrót danego modelu maszyny. Typowe wartości kroków na obrót w silnikach krokowych hybrydowych to 100 (krok o rozdzielczości 3.6\degree), 200 (1.8\degree) i 400 (0.9\degree). Aby uzyskać większą rozdzielczość kroku, należy zastosować sterowanie półkrokowe lub mikrokrokowe.

\tocLessLeftNorm{Sterowanie półkrokowe}

Tryb półkrokowy jest połączeniem sterowania falowego i pełnokrokowego. Rozdzielczość obrotowa względem poprzednich trybów jest dwa razy większa, dlatego że uzwojenia naprzemiennie zasilane są pojedynczo/ oba jednocześnie.

\insertImgSetSize{grafiki/stepper_sterowanie_polkrokowe.eps}
		{100}
		{Sekwencja zasilania uzwojeń przy sterowaniu półkrokowym (obrót wirnika z rys. \ref{grafiki/stepper_simple.png} według wskazówek zegara).}
		{oprWlasne}
		
Wadą takiego rozwiązania jest zmienny moment obrotowy- co drugi krok będzie "słabszy". Może to spowodować szarpanie wirnika przy dużych obrotach. Przy niskich prędkościach i obciążeniach natomiast, efekt ten rzadko występuje i można uzyskać polepszoną płynność ruchu, w porównaniu do poprzednich trybów. Aby uniknąć efektu zmiennego momentu, można zastosować sterowanie półkrokowe ze zmiennym prądem (rys. \ref{grafiki/stepper_sterowanie_polkrokowe_zmienne.eps}). Na rysunku widać że w momentach kiedy różnica potencjałów występuje tylko na jednym z uzwojeń, jest ona dwa razy większa niż w przypadku kiedy zasilane są obie cewki. Prąd płynący w uzwojeniach jest proporcjonalny do przyłożonego do te nich napięcia, dlatego moment się wyrównuje i silnik nie jest narażony na szarpanie. Rozwiązanie te także nie pozostaje bez wad- Zakładając że dla momentu w czasie, w którym przez oba uzwojenia płynie prąd, wartość tego prądu jest maksymalna dla danego modelu silnika w każdym z uzwojeń, oznacza to że w chwilach kiedy kiedy napięcie jest podawane na tylko jedno uzwojenie, prąd w nim płynący jest większy od nominalnego (tak jak na rys. \ref{grafiki/stepper_sterowanie_polkrokowe_zmienne.eps}). To naraża silnik na nadmierne grzanie, a nawet na uszkodzenie. Z drugiej strony można założyć że nominalny prąd ma płynąć w momencie zasilania jednego uzwojenia, a dwa razy mniejszy przy zasilaniu dwóch. Ten tryb jest bezpieczny, ale optymalny moment obrotowy nie jest uzyskiwany.

\insertImgSetSize{grafiki/stepper_sterowanie_polkrokowe_zmienne.eps}
		{100}
		{Sekwencja zasilania uzwojeń przy sterowaniu półkrokowym ze zmiennym prądem (obrót wirnika z rys. \ref{grafiki/stepper_simple.png} według wskazówek zegara).}
		{oprWlasne}

\clearpage


\subsection{Silniki prądu stałego (DC)}

\subsubsection{Budowa}

\subsubsection{Zasada działania}

\subsubsection{Sterowanie}

\clearpage


\subsection{Silnik bezszczotkowy prądu stałego (BLDC)}



\clearpage

