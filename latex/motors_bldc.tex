\subsection{Silniki synchroniczne z magnesami trwałymi}

Konstrukcja silnika synchronicznego z magnesami trwałymi jest odwrotnością silnika komutatorowego z magnesem trwałym: uzwojenia znajdują się w stojanie, a wirnik jest wykonany z odpowiednio ukształtowanego magnesu. Maszyny te są czasami  zaliczane do grupy silników prądu stałego, ale w praktyce pradem stałym zasilany jest jedynie falownik sterujący silnikiem. 

\insertImgSetSize{grafiki/bldc_przekroj.jpg}
		{50}
		{Przekrój typowego silnika trójfazowego z magnesami trwałymi: a) z wirnikiem wewnętrznym, b) z wirnikiem zewnętrznym, 1- jarzmo stojana (twornika), 2- zęby stojana, 3- żłobek z uzwojeniami, 4- wirnik, 5- magnes trwały}
		{krykowski}
		
Silniki synchroniczne z magnesami trwałymi są produkowane w dwóch odmianach konstrukcyjnych, różniących się rozkładem indukcji w szczelinie powietrznej oraz przebiegiem prądu i SEM (rys. \ref{grafiki/przebiegi_bldc_pmsm.png}):

\begin{easylist}
	& PMBLDC ({\it Permanent Magnet Brushless DC Motor})-- Silnik z trapezoidalnym przebiegiem SEM, zasilany prądem o przebiegu w przybliżeniu prostokątnym,
	& PMSM ({\it Permanent Magnet Synchronous Motor})-- Silnik z sinusoidalnym przebiegiem SEM, zasilany prądem o przebiegu sinusoidalnym.
	\\
\end{easylist} 
		
Silniki BLDC są konstrukcyjnie bardzo podobne do silników krokowych- można przyjąć że silniki BLDC są pewną podgrupą silników krokowych o kroku 180\degree (silniki 2-fazowe) lub 120\degree (3-fazowe). Kryterium odróżniające silniki krokowe od BLDC może być takie, że w większości aplikacji dla poprawnej pracy silnika BLDC jego sterownik musi dysponować sygnałem zwrotnym o aktualnym położeniu kątowym wirnika z czujników Hall'a. Silniki krokowe natomiast mogą pracować w otwartej pętli sprzężenia zwrotnego. \\

PMSM wymagają ciągłego pomiaru położenia wirnika np. przy pomocy enkodera, który im ma większą rozdzielczość, tym dokładniejsze sterowanie umożliwia.

\insertImgSetSize{grafiki/przebiegi_bldc_pmsm.png}
		{100}
		{Przebieg indukcji, SEM i prądu jednej fazy silnika o magnesach trwałych: a) BLDC, b) PMSM}
		{zawirski_pmsm} 

Z tego powodu najczęściej spotyka się rozwiązanie, w którym czujnik położenia wirnika jest zintegrowany z silnikiem we wspólnej obudowie. Wymaganą siłę elektromotoryczną uzyskuje się przez odpowiednie rozmieszczenie uzwojeń w żłobkach stojana i ukształtowanie rozkładu pola magnesów w szczelinie powietrznej. Obie wersje konstrukcyjne w teorii dają stałą moc, a więc stały moment wzdłuż szczeliny powietrznej. W rzeczywistości jednak moment silnika zawiera pulsacje powodujące nierównomierny ruch wirnika, zwłaszcza w przedziale bardzo małych prędkości \cite{zawirski_pmsm}. \\
		
Uzwojenia silnika synchronicznego z magnesami trwałymi dla maszyny 3-fazowej są układane w gwiazdę (rys. \ref{grafiki/bldc_gwiazda.jpg}). Prąd z układu sterującego jest doprowadzany do wyprowadzeń A1, B1 i C1 podczas gdy pozostałe końce uzwojeń (A2, B2, C2) są ze sobą zwarte. Oznacza to że w tym rodzaju maszyny, prąd przepływa zawsze przez co najmniej dwa uzwojenia, w jednym z dwóch możliwych kierunków (np. kiedy wyższy potencjał jest na C1, a niższy na A1 wtedy od C1 do A1). \\
		
Zmiany momentu obrotowego w funkcji kąta obrotu są w silniku trójfazowym niewielkie i można je jeszcze bardziej zredukować stosując impulsy prądowe o kształcie trapezowym. 
Tak jak zaprezentowano na rys. \ref{grafiki/bldc_przekroj.jpg} silniki bezszczotkowe mogą mieć różną konstrukcję wirnika. W przypadku budowy klasycznej wirnik w kształcie walca jest umieszczony wewnątrz stojana z uzwojeniami. W silniku z wirnikiem zewnętrznym magnetyczny wirnik ma kształt kubka i obraca się wokół nieruchomego stojana. 

\insertImgSetSize{grafiki/bldc_gwiazda.jpg}
		{50}
		{Konfiguracja uzwojeń trójfazowego silnika synchronicznego w gwiazdę}
		{przepiorkowski}

Do głównych zalet silników BLDC i PMSM należy zaliczyć to że zjawisko rezonansu mechanicznego jest o wiele mniejsze niż w przypadku silników krokowych, oraz mogą one uzyskiwać prędkości obrotowe o wiele większe te pierwsze (powyżej 10000 RPM). Ponadto, przy tej samej wartości momentu obrotowego silniki te mogą być mniejsze i lżejsze od krokowego. Z racji tego że nie posiadają komutatora, a co za tym idzie szczotek, ich żywotność jest bardzo długa i jest praktycznie ograniczona żywotnością mechaniki układu napędowego. Największą niedogodnością jest natomiast to, że z silnikiemi BLDC i PMSM nie da się w praktyce pracować w otwartej pętli sprzężenia zwrotnego.



\clearpage

