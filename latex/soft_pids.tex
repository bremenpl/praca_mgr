\subsection{Procedury sterowania wysokopoziomowego}

Sterowanie wyższego poziomu w kontrolerze polega na manipulacji, czy też operacjach na danych, które nie mają jeszcze bezpośredniego wpływu prąd w uzwojeniach czy pozycję kątową wału silnika. Odzwierciedlają one jedynie różne wielkości, które następnie są przekazywane do niższej warstwy oprogramowania, która ma steruje silnikami (\ref{sss:lowlevel}). \\

Procedury sterowania wysokopoziomowego dla obu rodzajów silników (krokowy i VCM) w większości są identyczne. Odmiennie wyglądają moduły sterowania prądem w uzwojeniach, ze względu na odmienną konstrukcję silników. Rys. \ref{grafiki/pid_vcm.pdf} prezentuje poglądowy diagram sterowania silnika VCM, a rys. \ref{grafiki/pid_stepper.pdf} silnika krokowego.

\clearpage

\insertImgSetSize{grafiki/pid_vcm.pdf}
	{200}
	{Schemat blokowy sterowania wysokopoziomowego silnikiem VCM, zaimplementowanego w kodzie sterownika}
	{oprWlasne}
	
\clearpage
	
\insertImgSetSize{grafiki/pid_stepper.pdf}
	{180}
	{Schemat blokowy sterowania wysokopoziomowego silnikiem krokowym, zaimplementowanego w kodzie sterownika}
	{oprWlasne}
	
\clearpage
	
\subsubsection{Regulatory pozycji i prędkości}
\label{sss:posvelregs}

W celu zapewnienia zadanej pozycji (kątowej w przypadku silnika krokowego i liniowej przy VCM) w sterowniku został zastosowany regulator pozycji typu PID-- proporcjonalno-całkująco-różniczkujący ({\it proportional–integral–derivative}). Jego zasadę działania wyraża wzór \ref{eq:pid1}.

\begin{equation} \label{eq:pid1}
	s(t) = K_p e(t) + K_i \int_0^t e(t)dt + K_d \frac{de(t)}{dt}
\end{equation}

Gdzie:

\begin{easylist}
	& $ s(t) $ : pozycja w danej chwili $ t $ $ [m] $,
	& $ e(t) $ : błąd pozycji $ e(t) = v(t) - v(t - 1) [m] $,
	& $ K_p  $ : współczynnik proporcjonalny $ [m] $,
	& $ K_i  $ : współczynnik całkujący $ [m \cdot s] $,
	& $ K_d  $ : współczynnik różniczkujący $ [\frac{m}{s}] $.
	\\
\end{easylist} 

Implementację algorytmu w kodzie mikrokontrolera przedstawia listing \ref{kody/pid_PosController.c}.

\insertCode{kody/pid_PosController.c}
	{C}
	{Kod regulatora pozycji zaimplementowany w  mikrokontrolerze, wywoływany w interwale $ dt = 0.001 s $}
	{mechsyscode}
	
Funkcja zawiera dodatkowo mechanizm zabezpieczający regulator przed zmagazynowaniem zbyt dużego błędu pozycji, co może spowodować przeregulowanie układu ({\it ,,Anti-Windup''}). Jako parametry wejściowe podawane są aktualna pozycja (odczytana z enkodera) i pozycja zadana. Wyjście regulatora pozycji jest następnie podawane na wejście regulatora prędkości (\ref{grafiki/pid_vcm.pdf}). \\

Regulator prędkości działa w ten sam sposób co regulator pozycji, z tą różnicą że sygnał wyjściowy z regulatora jest wielkością tego samego typu co sygnał wejściowy (w uproszczeniu, bo tylko regulator typu P daje na wyjściu sygnał tej samej wielkości co na wejściu). Suma (lub różnica w zależności od znaku zmiennej zwracanej z funkcji) tych dwóch sygnałów tworzy regulowaną prędkość zadaną. Zastosowanie oddzielnych regulatorów dla pozycji i prędkości pozwala w bardzo precyzyjny sposób konfigurować pracę danego silnika przy pomocy współczynników $ K_p $, $ K_i $ i $ K_d $ ({\it tunning}). Kolejne bloki różnią się w zależności od zastosowanego silnika.

\subsubsection{Sterowanie prędkością kątową silnika krokowego}

W modelu sterowania, który został zaimplementowany w oprogramowaniu sterownika, parametrami decydującymi o aktualnej prędkości kątowej silnika krokowego są:

\begin{easylist}
	& częstotliwość komutacji,
	& tryb pracy (,,bieg'').
\end{easylist} 

\insertImgSetSize{grafiki/freq_gear_stepper.eps}
	{100}
	{Algorytm obliczający częstotliwość komutacji i bieg dla zadanej prędkości obrotowej wału}
	{oprWlasne}

Wartości wymienionych parametrów są automatycznie obliczane w pętli sterowania dla zadanej prędkości obrotowej. W celu  wstępnej inicjalizacji algorytmu, należy podać następujące parametry:

\begin{easylist}
	& ilość pełnych kroków na obrót w danym modelu silnika,
	& ilość biegów (maksymalna dopuszczalna rozdzielczość pracy mikrokrokowej),
	& maksymalna częstotliwość komutacji $ [Hz] $,
	& maksymalna użyteczna prędkość kątowa dla danego silnika $ [RPM] $.
	\\
\end{easylist} 

<dokonczyc>

\subsubsection{Regulator prądu silnika VCM}

Wartość prądu zadanego dla uzwojenia maszyny VCM uzyskuje się poprzez jego estymatę, biorąc pod uwagę właściwości fizyczne układu napędowego (blok {\it VCM CURRENT ESTIMATOR} na schemacie \ref{grafiki/pid_vcm.pdf}). Na wejście estymatora trafia wartość absolutnego (względem obudowy), aktualnego wysuwu cylindra silnika. Zadana wartość wysuwu jest uzyskiwana poprzez pomnożenie stałej czasowej układu przez scałkowaną po czasie prędkość z jaką porusza się cylinder (\ref{eq:pid2}).

\begin{equation} \label{eq:pid2}
	stroke(t) = \int_0^t v(t) dt
\end{equation}

Do obliczenia zadanego prądu dla silnika VCM należy się posłużyć modelem fizycznym z rys. \ref{grafiki/sprezyna_vcm.eps}.

\insertImgSetSize{grafiki/sprezyna_vcm.eps}
	{50}
	{Teoretyczny model układu napędowego modułu z silnikiem VCM}
	{oprWlasne}
	
Obiekt $ A $ reprezentuje część ruchomą głowicy, czyli wszystkie komponenty znajdujące się po stronie cylindra (rys. \ref{grafiki/glowica.png}). $ B $ to część nieruchoma, po stronie której znajduje się obudowa silnika. W pierwszej kolejności należy obliczyć działającą na ciało $ A $ siłę sprężyny $ F_s $ (\ref{eq:pid3}):

\begin{equation} \label{eq:pid3}
	F_s = F_{s0} + R_s \cdot S \quad [ N = N + \frac{N}{m} \cdot m ]
\end{equation}

\begin{easylist}
	& $ F_{s0} $ : siła początkowa sprężyny $ [N] $,
	& $ R_s $ : stała sprężyny $ [\frac{N}{m}] $,
	& $ S $ : wysuw cylindra $ [m] $.
	\\
\end{easylist} 

Następnie należy obliczyć siłę $ F_m $ z jaką oddziałuje na ciało grawitacja (\ref{eq:pid4}):

\begin{equation} \label{eq:pid4}
	F_m = m_A \cdot g \cdot \sin \alpha \quad [ N = kg \cdot \frac{m}{s^2} \cdot 1 ]
\end{equation}

\begin{easylist}
	& $ m_A $ : sumaryczna masa części ruchomej głowicy $ [kg] $,
	& $ g $ : przyspieszenie ziemskie $ [\frac{m}{s^2}] $,
	& $ \alpha $ : kąt ustawienia głowicy względem płaszczyzny (normalnie 90\degree{}) $ [deg] $.
	\\
\end{easylist} 

Potem liczona jest wypadkowa siła $ F $ (\ref{eq:pid5}) działająca na ciało $ A $. Im cięższa jest część ruchoma głowicy, tym bardziej jest wysunięty cylinder w stanie spoczynku (bezprądowym). 

\begin{equation} \label{eq:pid5}
	F = F_s - F_m \quad [ N = N - N ]
\end{equation}

Ostatnim krokiem jest odczytanie stałej silnika VCM ($ F_c $ -- {\it force constant}) z charakterystyki dostarczonej przez producenta (np. \ref{grafiki/vcm_force_current_plot.eps}) i obliczenie prądu w uzwojeniu dla wymaganego wysuwu cylindra (\ref{eq:pid5}):

\begin{equation} \label{eq:pid6}
	I = \frac{F}{F_c} \quad [ A = \frac{N}{\frac{N}{A}} ]
\end{equation}

\insertImgSetSize{grafiki/vcm_current.eps}
	{100}
	{Przebieg teoretycznego prądu (oś Y) wymaganego w uzwojeniu do wysuwu cylindra na zadaną odległość (oś X)}
	{oprWlasne}
	
Na rys. \ref{grafiki/vcm_current.eps} widać wynik symulacji w postaci prądu w funkcji wysuwu cylindra. Obliczenia zostały przeprowadzone przy rzeczywistych wartościach parametrów układu napędowego. Wykres wskazuje na to, że w stanie spoczynku część ruchoma głowicy jest wysunięta na odległość ok. 6.7 mm. Charakterystyka ma charakter liniowy w swojej środkowej części, ze względu na kształt charakterystyki prądowej silnika wprowadzonej do algorytmu jako parametr (\ref{grafiki/vcm_force.eps}).

\insertImgSetSize{grafiki/vcm_force.eps}
	{100}
	{Charakterystyka prądowa zastosowanego silnika VCM odtworzona w programie używając charakterystyki z karty katalogowej}
	{oprWlasne}

Wyznaczony w zaprezentowany sposób prąd zadany trafia do właściwego regulatora prądu typu PI. jego realizacja różni się od poprzednich regulatorów tym, że nie ma już członu różniczkującego, który w przypadku regulacji prądu nie dał by wielu korzyści. Wartość prądu jest następnie zamieniana na odpowiednie wypełnienie impulsów w uzwojeniu-- więcej informacji na ten temat jest zawarte w sekcji \ref{sss:lowlevel}.

\subsubsection{Regulator prądu silnika krokowego}

Regulator prądu w przypadku silnika krokowego próbuje przez cały okres pracy maszyny utrzymywać w uzwojeniach taki prąd, aby spełnione było równanie \ref{eq:stepper2}. W tym celu, sygnałem zadanym dla regulatora jest zawsze maksymalny prąd dla pojedynczego uzwojenia wprowadzony jako parametr do systemu przez użytkownika ({\it CUR MAX SET} na schemacie \ref{grafiki/pid_stepper.pdf}). \\

Prąd płynący w uzwojeniach nie ma bezpośredniego wpływu na pozycję i prędkość kątową wirnika, a jedynie na moment obrotowy.















\clearpage