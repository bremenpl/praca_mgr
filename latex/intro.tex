\section*{Wprowadzenie}
Chcąc zaprojektować system napędowy dla określonej aplikacji należy wziąć pod uwagę bardzo wiele czynników. Po odpowiedzeniu sobie na 2 podstawowe pytania jednak, ilość zmiennych ulega zmniejszeniu dzięki temu że odpowiedzi nadają kierunek kolejnym rozważaniom. Te pytania to: Ile chcemy zapłacić za system, oraz ile inżynierskiej pracy (której lwia część to oprogramowanie) jesteśmy w stanie wykonać sami. Skłaniając się coraz bardziej w stronę gotowych rozwiązań, szybko zauważymy że ilość środków finansowych jakie jesteśmy zmuszeni wyłożyć jest bardzo duża. Jest to cena za łatwość i szybkość wdrożenia danego sterownika do projektu. Ceny takich systemów bardzo rzadko mają odzwierciedlenie w zastosowanym w ich strukturze {\it hardwarze}, jest to raczej cena za {\it know-how} producenta, wiedzę ekspercką, wsparcie techniczne oraz dziesiątki roboczogodzin programistów i elektroników tworzących produkt. W przypadku posiadania dużych środków takie rozwiązanie wydaje się być najbardziej odpowiednie. Ma ono jednak jedną znaczącą wadę (pomijając cenę oczywiście)- Pomimo tego że producenci sterowników do silników prześcigają się w ilości posiadanych cech czy technologii, to korzystając z takiego urządzenia nijako jesteśmy zmuszeni z korzystania z zamkniętego, ograniczonego systemu. System taki może lepiej lub gorzej wpasowywać się w ramy naszej aplikacji. Rzadko kiedy wpasowuje się idealnie, a najczęściej brakuje mu jakichś właściwości, lub jest przewymiarowany. \

Drugą opcją jest zakup taniego zintegrowanego chipu, który po raz kolejny w zależności od wyłożonych środków (w o wiele mniejszej skali), zawiera w sobie mniej lub więcej logiki i czasami końcówki mocy. Taki układ można także zbudować z elementów dyskretnych, co nie obniża znacząco kosztów, lecz daje możliwość większej kontroli nad parametrami układu przez to że nie używamy elementów "czarnych skrzynek". Wady takiego rozwiązania są takie, że zintegrowane sterowniki są projektowane pod zastosowanie z konkretnym rodzajem silnika (np tylko do silników krokowych) i służą do sterowania silnikami małej mocy jeśli mają zintegrowane końcówki. W przypadku stosowania układów tego typu oraz dyskretnych, projektant jest zmuszony zaprojektować obwód drukowany oraz napisać bardzo dużą ilość kodu zawierającego wiele skomplikowanych procedur, które muszą działać w czasie rzeczywistym. \

W przypadku, kiedy ilość produkowanych maszyn współpracującymi ze sterownikami do silników jest duża, posiadając odpowiednią wiedzę i umiejętności, bardziej opłaca się finansowo zaprojektować rozwiązanie własne, "od zera". Ponadto, proces tworzenia tego typu systemu dostarcza ogromną ilość wiedzy praktycznej, której zdobycie inaczej było by bardzo utrudnione.
  
\subsection*{Cel pracy}
Zbudować sterownik do sterowania różnego typu silników (krokowych, {\it BLDC}, liniowych i {\it DC}), z możliwością dostosowywania parametrów, używając aplikacji PC. Aplikacja ma pozwalać strojenie działających w sterowniku regulatorów, wybierać tryby pracy i badać działanie systemu w czasie rzeczywistym. Urządzenie ma zostać użyte w maszynach przemysłowych służących do produkcji urządzeń elektronicznych.

\subsection*{Zakres pracy}
Przegląd i analiza rozwiązań technologicznych, układów elektronicznych i oprogramowania stosowanych w sterownikach silników oferowanych przez wybranych producentów. Projekt uniwersalnego, kompaktowego sterownika do silników {\it DC} (prądu stałego), {\it BLDC (Brushless DC Motor)}, krokowych oraz liniowych {\it VCM (Voice Coil Motor}, inaczej {\it VCA - Voice Coil Actuator}) wraz z opisem sposobów regulacji i algorytmów zawierających elementy {\it AI}, które zostaną zaimplementowane w nowoczesnym 32 bitowym mikrokontrolerze z rodziny {\it ARM}. Opracowanie aplikacji na PC umożliwiającej dostrajanie parametrów dla konkretnego silnika oraz analizowanie jego pracy. Wykonanie, uruchomienie i badania laboratoryjne prototypu urządzenia. Prezentacja przykładowego zastosowania sterownika w rzeczywistym układzie.
