\section*{Podsumowanie}

Będący tematem pracy \paperTitleSmallPl{} nie został jeszcze ukończony, ale jest już funkcjonalny. \linebreak Ze sprzętowego punktu widzenia kontroler jest gotowy dla głowicy układającej do automatów typu Pick and Place. W przypadku potrzeby zastosowania go w innej aplikacji, należy zaprojektować nową płytkę bazując na podzespołach zastosowanych w tym projekcie. \\

Pod względem oprogramowania wszystkie bazowe moduły sterowanie są napisane. Następnym krokiem będzie dodanie procedur dla maszyn BLDC/ PMSM \linebreak i komutatorowych. Dużej zmiany w oprogramowaniu wymaga moduł wyższego poziomu, wykonujący poszczególne funkcje (np. podnoszenie komponentu). Są to funkcjonalności charakterystyczne dla danej aplikacji i nie powinny być częścią kodu bazowego uniwersalnego sterownika. Docelowo należy zaprojektować dedykowany język skryptowy, pozwalający użytkownikowi na pisanie kodu wysokiego poziomu i zapisywanie go w pamięci kontrolera. Sterownik powinien potrafić interpretować skrypt i wykonywać sekwencyjnie zapisane w nim instrukcje. \\

W celu napisania skryptu potrzebne jest specjalne środowisko: dedykowana aplikacja napisana na platformę PC. Program musi także umożliwiać konfigurację parametrów danego sterownika, takich jak np. rodzaj podłączonych silników czy enkoderów. Dodatkowo musi pozwalać na dostrajanie regulatorów i filtrów oraz odczytywanie parametrów chwilowych (np. prądy w uzwojeniach) w czasie rzeczywistym. Moduł logger'a (sekcja \ref{ss:hwrs232}) nie jest w stanie zapewnić takiej funkcjonalności, gdyż jego priorytet jest bardzo niski i nie można go zwiększyć nie zaburzając pracy sterownika. Z tego powodu potrzebny jest dodatkowy moduł pozwalający na zapis danych do pamięci RAM mikrokontrolera, a następnie wysłanie ich po zakończeniu skryptu. \\

Ostatnim elementem (po zakończeniu zadań o wyższym priorytecie) potrzebnym w projekcie jest wyposażenie aplikacji konfiguracyjnej i samego sterownika w elementy sztucznej inteligencji. Zastosowanie np. sieci neuronowej mogło by pozwolić na automatyczne dostosowanie optymalnych współczynników dla regulatorów.


\clearpage





