\subsection{Tryby pracy}

Obserwując kierunek przepływu energii w maszynie elektrycznej, można wyróżnić jej dwa tryby pracy: silnikową oraz generatorową. W stanach ustalonych układu napędowego zjawiska te można zobrazować za pomocą charakterystyk mechanicznych (rys. \ref{grafiki/ch_statyczne.eps}), pokazujących zależność momentu $ M $ od prędkości $ \Omega $. Kształt charakterystyk zależy przede wszystkim od struktury i parametrów układów regulacyjnych i pomiarowych, a w pewnym zakresie również od charakterystyk przekształtnika oraz od typu i parametrów silnika \cite{zawirski}.

\insertImgSetSize{grafiki/ch_statyczne.eps}
	{100}
	{Charakterystyki statyczne silnika: $ \Omega $ - prędkość obrotowa, $ M $ - moment obrotowy}
	{oprWlasne}
	
Moc na wale maszyny jest wyrażana wzorem:

\begin{equation} \label{eq:chs1}
	P = \Omega M
\end{equation}

Kiedy jest dodatnia (kwadrant I i III) oznacza to, że maszyna pobiera ją ze źródła napięcia i pracuje jako silnik napędzając element będący obciążeniem. W przypadku kiedy moc jest ujemna (kwadrant II i IV) silnik hamuje obciążenie pracując jako prądnica (praca generatorowa), która poprzez przekształtnik oddaje energię do źródła \cite{ecn_luty_2012}. \\

W zależności od topologii układu przekształtnikowego i zastosowanych algorytmów sterowania maszyna elektryczna może mieć możliwość pracy tylko w części kwadrantów lub we wszystkich . 








