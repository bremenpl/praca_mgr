\subsection{Algorytmy enkodera magnetycznego}
\label{ss:encalg}

W odróżnieniu do enkodera liniowego, dane odczytywane ze skonstruowanego modułu enkodera magnetycznego bazującego na układzie AS5048A (rys. \ref{grafiki/AS5048A_diagram.eps}) są obarczone błędami związanymi z nieliniowością układu i typem enkodera. Bez zastosowania żadnych algorytmów uśredniających odczytaną pozycję kątową, dokładność systemu była by marna, a sam układ napędowy działał by niestabilnie. \\

Aby umożliwić pracę enkodera z rozdzielczością do 0.05\degree{} w sterowniku zostały zaimplementowane następujące algorytmy:

\begin{easylist}
	& uśrednianie bazujące na średniej arytmetycznej,
	& filtr Kalmana (opracowany na podstawie \cite{kalman} i \cite{forbot}),
	& filtr dolnoprzepustowy jednobiegunowy, rekursywny (opracowany na podstawie \cite{smith}).
	\\
\end{easylist} 

\subsubsection{Uśrednianie}

Pierwszy z wymienionych algorytmów wykonywany jest zawsze. Pozostałe dwa natomiast są uruchamiane naprzemiennie, w zależności od konfiguracji użytkownika. Uśrednianie działa na prostej zasadzie. Wykonywanych jest kilka odczytów pozycji z enkodera (ilość odczytów zależy od konfiguracji, najczęściej cztery), z których następnie obliczana jest średnia arytmetyczna (wzór \ref{eq:alg1}). W kolejnym kroku odczyt jest normalizowany (wzór \ref{eq:alg2})-- mechanizm jest wymagany, kiedy pozycja waha się między 359-tym, a zerowym stopniem.

\begin{equation} \label{eq:alg1}
	pos = \frac{a_0 + a_1 + a_2 + ... + a_n}{n} 
\end{equation}

\begin{equation} \label{eq:alg2}
pos = \left\{
  \begin{array}{ll}
    pos + 360 & : pos < 0 \\
    pos - 360 & : pos > 360
  \end{array}
\right.
\end{equation}

\begin{easylist}
	& $ pos $ : absolutna pozycja enkodera [deg],
	& $ n $ : ilość próbek do uśrednienia.
	\\
\end{easylist} 

Zbyt duża ilość odczytów nie jest opłacalna, gdyż błąd odczytu przestaje znacząco maleć (wzór \ref{eq:alg3}).

\begin{equation} \label{eq:alg3}
	x = \frac{1}{\sqrt{n}}
\end{equation}

\begin{easylist}
	& $ x $ : szum enkodera (max. 0.06\degree{}),
	& $ n $ : ilość próbek do uśrednienia.
\end{easylist} 

\subsubsection{Filtr Kalmana}

Zastosowany algorytm jest metodą filtracji dynamicznej. Posługując się tym narzędziem, można wyznaczyć pomiarowo niedostępne zmienne jedynie na podstawie bieżących wartości wielkości pomiarowo dostępnych oraz znajomości modelu matematycznego łączącego ze sobą obydwie te grupy pomiarów. \\

Proces można przedstawić za pomocą dyskretnego modelu w przestrzeni stanu:

\begin{equation} \label{eq:alg4}
	x(t + 1) = Ax(t) + Bu(t) + v(t)
\end{equation}

\begin{equation} \label{eq:alg5}
	y(t) = Cx(t) + w(t) 
\end{equation}

gdzie:
\begin{easylist}
	& $ x(t) $ : stan w chwili czasu $ t = 0, 1,... $,
	& $ y(t) $ : wyjście układu,
	& $ A $ : macierz stanu,
	& $ B $ : macierz wejścia,
	& $ C $ : macierz wyjścia,
	& $ v(t) $ : szum procesowy,
	& $ w(t) $ : szum pomiarowy.
\end{easylist} 

\insertImgSetSize{grafiki/przebieg_kf.eps}
	{110}
	{Przebieg czasowy prezentujący pozycję wału silnika krokowego: zadaną, mierzoną i mierzoną przefiltrowaną przez {\it KF}}
	{oprWlasne}

Zastosowany filtr jest dwuwymiarowy. Oznacza to że estymacji są poddawane dwa parametry: pozycja i prędkość. Model stanowy wygląda następująco:

\begin{equation} \label{eq:alg6}
	x = \begin{bmatrix}
       	pos \\ 
       	vel
     	\end{bmatrix}
\end{equation}

\begin{equation} \label{eq:alg7}
	A = \begin{bmatrix}
       	1 & -dt \\ 
       	0 & 1
     	\end{bmatrix}
\end{equation}

\begin{equation} \label{eq:alg8}
	B = \begin{bmatrix}
       	dt \\ 
       	0
     	\end{bmatrix}
\end{equation}

\begin{equation} \label{eq:alg9}
	C = \begin{bmatrix}
       	1 & 0 
     	\end{bmatrix}
\end{equation}

Implementacja dwuwymiarowa zapewnia dokładniejszą estymatę pozycji niż w przypadku, kiedy pod uwagę brane byłoby tylko położenie. Jeszcze lepsze wyniki możnaby otrzymać stosując estymatę trójwymiarową (dodatkowo dodać parametr przyspieszenia). Niestety operacje na macierzach 3 x 3 pochłaniały za dużo zasobów procesora i z tego rozwiązania w sterowniku zrezygnowano. \\

Rys. \ref{grafiki/przebieg_kf.eps} prezentuje różne pozycje kątowe w funkcji czasu odczytane ze sterownika w trybie CSV (patrz sekcja \ref{ss:hwrs232}). Wstępnie widać, że algorytm działa poprawnie, lecz jego zalety i wady można dokładniej zaobserwować na przebiegu z rys. \ref{grafiki/przebieg_kf_zoom.eps}.

\insertImgSetSize{grafiki/przebieg_kf_zoom.eps}
	{110}
	{Przebieg czasowy (powiększenie przy impulsie skokowym) prezentujący pozycję wału silnika krokowego: zadaną, mierzoną i mierzoną przefiltrowaną przez {\it KF}}
	{oprWlasne}
	
Po doświadczalnym dostosowaniu parametrów $ v $ i $ w $ algorytm tłumi zakłócenia bardzo dobrze. Charakterystyczną cechą filtru Kalmana jest jednak to, że wraz ze wzrostem poziomu filtracji rośnie także opóźnienie sygnału. Parametr ten diametralnie wpływa na działanie regulatorów pozycji i prędkości (sekcja \ref{sss:posvelregs}). Pomimo tego, że estymowana pozycja jest bardzo blisko zadanej, to czas potrzebny na jej osiągnięcie (w końcowym odcinku) jest zbyt długi. Z tego powodu został zaimplementowany dodatkowy, szybszy algorytm stosowany w czasowo krytycznych sytuacjach.

\clearpage

\subsubsection{Filtr dolnoprzepustowy}

Drugim algorytmem stosowanym w celu minimalizacji wpływu szumów towarzyszących odczytywanej pozycji kątowej z enkodera magnetycznego jest rekursywny (lub też {\it NOI}-- o nieskończonej odpowiedzi impulsowej), jednobiegunowy filtr dolnoprzepustowy. Zastosowane zostały jedynie dwa współczynniki rekursywne $ a_0 $ i $ b1 $:

\begin{equation} \label{eq:alg10}
	a_0 = 1 - x
\end{equation}

\begin{equation} \label{eq:alg11}
	b_1 = x
\end{equation}

Współczynnik $ x $ wskazuje jakie jest opadanie przebiegu mierzone między kolejnymi próbkami. Wartość ta jest obliczana na podstawie stałej czasowej filtru:

\begin{equation} \label{eq:alg12}
	x = \exp^{-2 \pi f_{CR}}
\end{equation}

$ f_{CR} $ jest częstotliwością odcięcia dobieraną w sterowniku doświadczalnie (obecnie 20 Hz). Kolejne próbki na wyjściu filtra obliczane są według wzoru \ref{eq:alg13}

\begin{equation} \label{eq:alg13}
	output(t) = input(t) a_0 + input(t - 1) b_1 
\end{equation}

\insertImgSetSize{grafiki/przebieg_lowpass_zoom.eps}
	{110}
	{Przebieg czasowy (powiększenie przy impulsie skokowym) prezentujący pozycję wału silnika krokowego: zadaną,   mierzoną nieprzefiltrowaną i mierzoną przefiltrowaną przez filtr dolnoprzepustowy trzeciego rzędu}
	{oprWlasne}
	
Rząd filtru jest ustalany przez użytkownika. Rys. \ref{grafiki/przebieg_lowpass_zoom.eps} prezentuje odczyt pozycji wału silnika krokowego. Algorytm ten działa szybciej od zaimplementowanego filtru Kalmana-- obrót o 18\degree{} w ok. 1 s przy zastosowaniu KF i ok. 170 ms przy filtrze dolnoprzepustowym trzeciego rzędu. Ponadto algorytm jest bardzo prosty i kod potrzebny do jego wykonania nie wymaga dużej ilości zasobów mikrokontrolera (listing \ref{kody/low_pass.c}).

\insertCode{kody/low_pass.c}
	{C}
	{Kod filtra dolnoprzepustowego wykonywany w pętli sterującej mikrokontrolera w interwale $ dt = 0.001 s $}
	{mechsyscode}
		   
Na chwilę obecną uśrednianie wraz z  filtrem dolnoprzepustowym są algorytmami dającymi najlepsze wyniki działania enkodera.















