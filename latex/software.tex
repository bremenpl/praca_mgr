\section{Oprogramowanie}
\label{s:oprogramowanie}

Kod wykonywany w mikroprocesorze sterującym został napisany w języku c i w nielicznych częściach w assemblerze. Jądrem całej architektury jest system operacyjny czasu rzeczywistego FreeRTOS™. Wszystkie procedury wysokiego poziomu wykonywane w kodzie są umieszczone w wątkach o odpowiednich priorytetach, które porozumiewają się między sobą za pomocą kolejek, semaforów oraz obszarów pamięci współdzielonej. Zadania najniższego poziomu są wykonywane w procedurach obsługi przerwań, np. komutacja lub wkładanie przychodzących ramek CAN do kolejki. \\

Rysunek \ref{grafiki/Soft_architecture.pdf} przedstawia poglądowy schemat blokowy architektury systemu. Pomarańczowymi blokami zostały oznaczone procedury wykonywane na poziomie sprzętowym, lub programowym na warstwie znajdującej się niżej od wątków systemu FreeRTOS™. Kolor zielony oznacza wątki/ procesy w systemie FreeRTOS™. Strzałki z paskami symbolizują komunikaty przychodzące do kontrolera ze świata zewnętrznego, a bez nich komunikaty wewnątrz systemowe. Dokładniejszy opis poszczególnych bloków znajduje się w korespondujących z nimi sekcjach.

\insertImgSetSize{grafiki/peryferia.png}
	{110}
	{Widok wstępnie skonfigurowanych peryferiów w programie mxCube firmy ST Microelectronics}
	{oprWlasne}
	
Do wygenerowania kodu zawierającego wstępną konfigurację peryferiów służy program mxCube firmy ST Microelectronics.
	
\insertImgSetSize{grafiki/Soft_architecture.pdf}
	{220}
	{Schemat blokowy architektury oprogramowania sterownika}
	{oprWlasne}
	
\clearpage

Jego możliwości konfiguracyjne są bardzo podstawowe, ale pozwala on w graficzny sposób zobrazować używane moduły i wyprowadzenia (rys. \ref{grafiki/peryferia.png}). Dla bardziej skomplikowanych algorytmów przed implementacją zostały wykonane symulacje w programie MATLAB, który posłużył także do zobrazowania danych.

\subsection{Komunikacja}

Gotowe do obsługi programowej w urządzeniu są gotowe dwie magistrale szeregowe (patrz \ref{sss:hardware_komunikacja}). W tej sekcji znajduje się opis zastosowania każdej z nich na obecnym poziomie rozwoju.

\subsubsection{Magistrala CAN}

Ze sprzętowego punktu widzenia, magistrala pracuje w protokole CAN 2.0B (rozszerzona ramka: 29 bitowy identyfikator i 64 bitowe pole danych) z prędkością 1 Mbit/s. Patrząc od strony oprogramowania używany protokół to TML CAN, który jest używany przed dotychczas używane w maszynach sterowniki firmy Technosoft. Jak zostało wcześniej wspomniane, projektowane urządzenie ma za zadanie zastąpić w maszynie dwa dotychczasowe, dlatego z punktu widzenia sieci są to dwa kontrolery (rys. \ref{grafiki/can_nodes.pdf}).

\insertImgSetSize{grafiki/can_nodes.pdf}
	{60}
	{Topologia sieci CAN. Sterownik przedstawia się jako dwa osobne urządzenia}
	{st}
	
W rzeczywistości kontroler jest tylko jeden. Mikrokontroler STM32F07VTG6 posiada dwie sprzętowe kolejki typu FIFO ({\it First In First Out}) do odbierania ramek CAN, które są współdzielone przez moduły CAN1 i CAN2. Urządzenie korzysta tylko z jednego peryferia, więc obie kolejki mogą być zastosowane przez nie. Filtry ramek przychodzących są skonfigurowane w taki sposób, aby przepuszczać tylko te ramki, które posiadają w swoim identyfikatorze numer osi Z lub R (odpowiednio 5 i 6). Po odebraniu wiadomości dla osi Z, ramka trafia do pierwszej kolejki FIFO, jeśli wiadomość jest dla osi R, to do drugiej. Taka architektura zapewnia bardzo dobre odseparowanie wirtualnych modułów odbiorczych sterowników już na poziomie sprzętowym. Bufory FIFO mają jednak ograniczoną pojemność-- mieszczą tylko 3 ramki. Z tego względu wiadomości do nich wpadające muszą być jak najszybciej odczytane i ulokowane w programowej kolejce o rozmiarze 32 ramek, w której mogą oczekiwać na interpretację przez dłuższy okres czasu. \\

Dostęp do zmiennych (obszarów pamięci) w sterownikach Technosoft'u odbywa się poprzez zapis do i odczyt z rejestrów. Architektura konstruowanego kontrolera nie wspiera w natywny sposób takiego programowania, dlatego w celu poprawnej pracy potrzebna jest dodatkowa warstwa (rys. \ref{grafiki/tml_can_trans.pdf}).

\insertImgSetSize{grafiki/tml_can_trans.pdf}
	{35}
	{Sposób tłumaczenia ramek TML CAN w kontrolerze. Przykładowy zapis prędkości ruchu w osi Z}
	{oprWlasne}
	
W warstwie tłumaczenia ramek TML dokonywanych jest wiele różnych operacji. W zależności od komunikatu może to być zapis do zmiennej lub np. wywołanie funkcji. W przykładzie zaprezentowanym na rysunku \ref{grafiki/tml_can_trans.pdf} zewnętrzne urządzenia wysyła do kontrolera wiadomość TML, która po przeskalowaniu ustawia prędkość ruchu w osi Z.

\subsubsection{Magistrala RS232}
	


\clearpage











\subsection{Realizacja zadań}

Wysunięcie się cylindra silnika VCM o odległość $ x $, zwrócenie wartości podciśnienia w głowicy czy obrót komponentu o 90\degree -- wszystko to jest przykładam wywołania funkcji, czy też realizacją zadania. Każda funkcja ma na wykonanie się określoną ilość czasu i jest zbudowana na zasadzie skryptu, który czeka na zajście konkretnych wydarzeń ({\it Events}), interpretuje je i wykonuje kolejne procedury w zależności od ich wyniku. W przypadku nie wykonania się którejkolwiek części skryptu w zadanym czasie, skrypt jest anulowany i do urządzenia wydającego komendę zostaje zwrócony kod błędu. System interpretuje obecnie 9 zdarzeń (listing \ref{kody/sys_triggers.h}).

\insertCode{kody/sys_triggers.h}
		   {C}
		   {Obiekt typu {\it enum} prezentujący obsługiwane w sterowniku wydarzenia}
		   {mechsyscode}

Zdarzenia mogą zachodzić sekwencyjnie lub równolegle. Skrypt może oczekiwać na zajście kilku skryptów (operatory {\it AND} i {\it OR}), gdzie każdy z nich także posiada ustalony czas na wykonanie się. \\

Rysunek \ref{grafiki/uml_homing.pdf} przedstawia przykładową procedurę (homing osi Z) w języku UML ({\it Unified Modeling Language}) wykonywaną przez system. W pierwszej kolejności zewnętrzne urządzenie (Kontroler systemowy automatu Pick and Place) wysyła poprzez magistralę CAN wywołanie funkcji Homing'u. Ramka trafia do kolejki odbiorczej CAN. Po interpretacji zostaje wybrany odpowiedni skrypt ({\it JobDoHoming}) i rozpoczyna się jego wykonywanie. \\

Pierwszym etapem działania skryptu jest zamknięcie zaworu podciśnienia. Następnie sprawdza na którym polu paska referencyjnego znajduje się część ruchoma głowicy (sekcja \ref{ss:encopt}). Jeśli jest to pole czarne (cylinder silnika VCM wsunięty), następuje wejście w pierwsze rozwidlenie w skrypcie (Black strip). Procedura homing'u służy do kalibracji pozycji. Zanim zostanie wykonana, absolutna pozycja cylindra nie jest znana. Po włączeniu zasilania przyjmuje się że pozycja w której aktualnie znajduje się cylinder to zero. Wiedząc tylko że cylinder jest w polu czarnym paska, następuje wysuw cylindra według określonej trajektorii (sekcja \ref{sss:trajgen}). Od razu po tym, funkcja oczekuje na zejście z czarnej strefy paska na białą (wiadomość {\it EXTI} na diagramie). Kiedy to nastąpi, zostaje ustalona absolutna pozycja cylindra. Od tej chwili można swobodnie sterować wysuwem głowicy. Na końcu cylinder wraca do ustalonej pozycji startowej.

\insertImgSetSize{grafiki/uml_homing.pdf}
	{130}
	{Diagram interakcji prezentujący wykonanie przez system procedury bazowania ({\it homing}) w osi Z}
	{oprWlasne}

Jeśli z jakiegoś powodu podczas uruchomienia procedury homing'u cylinder nie jest wsunięty (np. rozciągnięta lub uszkodzona sprężyna) i głowica znajduje się nad polem białym, najpierw następuje jej wsunięcie, aby homing zawsze następował \linebreak w ten sam sposób (zejście z pola czarnego na białe). Potwierdzenie wykonania zadania jest wysyłane do urządzenia wywołującego funkcję, w tym wypadku do kontrolera systemu. \\

Wszystkie zadania obecnie obsługiwane przez sterownik są wykonywane na prezentowanej na rys. \ref{grafiki/uml_homing.pdf} zasadzie (np. podniesienie komponentu, położenie go lub obrócenie). 



















\clearpage