\subsection{Silnik komutatorowy (DC)}

Silniki komutatorowe są wynalazkiem stosunkowo starym i pomimo swoich wad są dalej powszechnie stosowane w produktach konsumenckich jak i w przemyśle, głównie ze względu na prostotę regulacji prędkości i momentu obrotowego.

\insertImg{grafiki/komutator.png}
		{Uproszczona budowa komutatora typowego silnika elektrycznego}
		{hyperphysics}
		
Komutator jest szeregiem miedzianych płytek umieszczonych na bocznej powierzchni wirnika, po którym ślizgają się (zazwyczaj) węglowe szczotki do których doprowadzone jest napięcie (rys. \ref{grafiki/komutator.png}). Zadaniem komutatora jest przełączanie kierunku przepływu prądu w uzwojeniach tak, aby oddziaływanie z polem magnetycznym stojana wprawiło wirnik w ruch obrotowy. Inaczej niż w przypadku silników VR, układ sterujący maszyną jest odciążony od zapewniania odpowiedniej komutacji, co sprawia że może on być o wiele prostszy w konstrukcji. 

\insertImgSetSize{grafiki/komutator_praca.png}
		{60}
		{Zasada działania silnika komutatorowego z magnesem trwałym}
		{hyperphysics}
		
Rysunek \ref{grafiki/komutator_praca.png} prezentuje zasadę działania silnika komutatorowego, w którym do wytworzenia pola magnetycznego stojana został użyty magnes trwały. Tak skonstruowany silnik to PMDC ({\it Permanent Magnet DC}). Kiedy prąd elektryczny $ I $ przepływa przez cewkę znajdującą się w polu magnetycznym $ B $, to siła magnetyczna $ F $ (Działająca prostopadle do ułożonych uzwojeń cewki i działającej indukcji)  wywołuje moment obrotowy wprawiający wirnik w ruch. Podaną zależność przedstawia wzór \ref{eq:silapola4}. W silniku tego rodzaju obroty zależą liniowo od napięcia zasilania, a moment obrotowy od natężenia prądu. Przy wzroście obciążenia spadają obroty, a rośnie moment obrotowy i pobór prądu. Zmiana kierunku obrotów jest uzyskiwana poprzez zmianę biegunowości zasilania. Prędkość obrotową można łatwo regulować w granicy od 5\% do 110\% prędkości znamionowej, z zachowaniem dużego momentu obrotowego. Dobrą stabilizację momentu obrotowego można uzyskać poprzez zasilanie silnika ze źródła prądowego. \\

Do wytworzenia pola magnetycznego stojana można użyć elektromagnesu, zamiast magnesu trwałego. Taki silnik posiada dwa uzwojenia: wirnika i stojana. W zależności od układu połączeń można otrzymać silnik:

\begin{easylist}
	& szeregowy,
	& bocznikowy,
	& szeregowo-bocznikowy.
\end{easylist}

\insertImgSetSize{grafiki/silniki_szereg_bocz.jpg}
		{70}
		{Możliwe konfiguracje uzwojeń wirnika i stojana: a) silnik szeregowy, b) bocznikowy, c) szeregowo-bocznikowy}
		{przepiorkowski}

Silnik szeregowy ma bardzo duży moment obrotowy i rozruchowy, ale prędkość obrotowa jest silnie uzależniona od obciążenia silnika. W warunkach pracy bez obciążenia, silnik szeregowy może rozpędzać się bez ograniczeń, aż do jego uszkodzenia (,,Rozbieganie się'' silnika). Silnik bocznikowy nie posiada tej wady- jego obroty są stałe i prawie niezależne od obciążenia. Niestety silniki bocznikowe zazwyczaj wymagają skomplikowanych układów płynnego rozruchu ograniczających prąd rozruchowy, a moment rozruchowy jest dużo mniejszy niż w modelach szeregowych. Wersja szeregowo-bocznikowa ma charakterystykę zbliżoną do szeregowego, a dodatkowe uzwojenie bocznikowe ogranicza maksymalne obroty przy pracy bez obciążenia. Regulacja obrotów jest możliwa poprzez zmianę napięcia zasilania. Kierunek obrotów jest stały, bez względu na biegunowość zasilania, co sprawia że maszynę można zasilać z sieci AC.

\clearpage

