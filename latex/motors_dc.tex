\subsection{Silnik komutatorowy (DC)}

Silniki komutatorowe pomimo swoich wad są powszechnie stosowane w produktach konsumenckich (jak i w przemyśle), głównie ze względu na prostotę regulacji prędkości i momentu obrotowego. 

\insertImgSetSize{grafiki/komutator_praca.png}
		{60}
		{Zasada działania silnika komutatorowego z magnesem trwałym}
		{hyperphysics}

Komutator jest szeregiem miedzianych płytek umieszczonych na bocznej powierzchni wirnika, po którym ślizgają się (zazwyczaj) węglowe szczotki do których doprowadzone jest napięcie (rys. \ref{grafiki/komutator_praca.png}). Zadaniem komutatora jest przełączanie kierunku przepływu prądu w uzwojeniach tak, aby oddziaływanie z polem magnetycznym stojana wprawiło wirnik w ruch obrotowy. Inaczej niż w przypadku silników VR, układ sterujący maszyną jest odciążony od zapewniania odpowiedniej komutacji, co sprawia, że może on być o wiele prostszy w konstrukcji. \\

Rysunek \ref{grafiki/komutator_praca.png} prezentuje zasadę działania silnika komutatorowego, w którym do wytworzenia pola magnetycznego stojana został użyty magnes trwały. Tak skonstruowany silnik to PMDC ({\it Permanent Magnet DC}). Kiedy prąd elektryczny $ I $ przepływa przez cewkę znajdującą się w polu magnetycznym $ B $, to siła magnetyczna $ F $ (Działająca prostopadle do ułożonych uzwojeń cewki i działającej indukcji)  wywołuje moment obrotowy wprawiający wirnik w ruch. W silniku tego rodzaju obroty zależą liniowo od napięcia zasilania, a moment obrotowy od natężenia prądu. Przy wzroście obciążenia spadają obroty, a rośnie moment obrotowy i pobór prądu. Zmiana kierunku obrotów jest uzyskiwana poprzez zmianę biegunowości zasilania. Prędkość obrotową można łatwo regulować w granicy od 5\% do 110\% prędkości znamionowej, z zachowaniem dużego momentu obrotowego. Dobrą stabilizację momentu obrotowego można uzyskać poprzez zasilanie silnika ze źródła prądowego. \\

Do wytworzenia pola magnetycznego stojana można użyć wzbudzenia elektromagnetycznego zamiast magnesu trwałego. Taki silnik posiada dwa uzwojenia: wirnika i stojana. W zależności od układu połączeń można otrzymać silnik:

\begin{easylist}
	& szeregowy,
	& bocznikowy,
	& szeregowo-bocznikowy.
\end{easylist}



