\subsection{Silniki komutatorowe (DC)}

Silniki komutatorowe są wynalazkiem stosunkowo starym i pomimo swoich wad są dalej powszechnie stosowane w produktach konsumenckich jak i w przemyśle, głównie ze względu na prostotę regulacji prędkości i momentu obrotowego.

\insertImg{grafiki/komutator.png}
		{Uproszczona budowa komutatora typowego silnika elektrycznego}
		{hyperphysics}
		
Komutator jest szeregiem miedzianych płytek umieszczonych na bocznej powierzchni wirnika, po którym ślizgają się (zazwyczaj) węglowe szczotki do których doprowadzone jest napięcie (rys. \ref{grafiki/komutator.png}). Zadaniem komutatora jest przełączanie kierunku przepływu prądu w uzwojeniach tak, aby oddziaływanie z polem magnetycznym stojana wprawiło wirnik w ruch obrotowy. Inaczej niż w przypadku silników VR, układ sterujący maszyną jest odciążony od zapewniania odpowiedniej komutacji, co sprawia że może on być o wiele prostszy w konstrukcji. 

\insertImgSetSize{grafiki/komutator_praca.png}
		{60}
		{Zasada działania silnika komutatorowego}
		{hyperphysics}
		
Rysunek \ref{grafiki/komutator_praca.png} prezentuje zasadę działania silnika komutatorowego. Kiedy prąd elektryczny $ I $ przepływa przez cewkę znajdującą się w polu magnetycznym $ B $, to siła magnetyczna $ F $ (Działająca prostopadle do ułożonych uzwojeń cewki i działającej indukcji)  wywołuję moment obrotowy wprawiający wirnik w ruch. Podaną zależność przedstawia wzór \ref{eq:silapola4}.

\clearpage

