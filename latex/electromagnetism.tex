\section{Elektromagnetyzm}

Aby zrozumieć zasadę działania większości silników elektrycznych, należy najpierw zaznajomić się z podstawowymi jednostkami występującymi w elektromagnetyzmie, dzięki którym silniki działają. Rozdział został opracowany na podstawie źródła \cite{jaszczuk}

\subsection{Natężenie pola magnetycznego}

Pole magnetyczne wytwarza się wokół przewodnika przez który płynie prąd elektryczny. Pole to charakteryzują wektory natężenia, które w każdym punkcie są styczne do okręgów linii sił pola. Środek każdego z okręgów jest zbieżny z osią geometryczną przewodu (przewodnika). Stosując regułę prawej dłoni (Rysunek \ref{grafiki/right_hand_rule.pdf}), palce wskazują zwrot wektorów, a kciuk zwrot i kierunek płynącego prądu.

\insertImg{grafiki/right_hand_rule.pdf}
	   {Reguła prawej dłoni - sposób wyznaczania zwrotu linii sił pola magnetycznego wokół przewodnika w którym jest prąd elektryczny}
	   {wikiRegPrawDlon}
	   
Do wyznaczenia natężenia pola magnetycznego w dowolnym punkcie na zewnątrz przewodu należy posłużyć się rysunkiem \ref{grafiki/zwrot_wektora_nat_pola.pdf}

\insertImg{grafiki/zwrot_wektora_nat_pola.pdf}
	   {Zwrot wektora natężenia pola magnetycznego w przestrzeni wokół przewodu przez który płynie prąd elektryczny (Zwrot za płaszczyznę rysunku od strony obserwatora)}
	   {oprWlasne}
	   
\begin{easylist}
	& K: punkt w przestrzeni wokół przewodnika,
	& g: Okrąg na którym leży punkt K,
	& r: promień okręgu g,
	& H: wektor natężenia pola.
	\\
\end{easylist} 

W przypadku linii pola, która przechodzi przez punkt K, przepływ przez powierzchnię o brzegu g równa się prądowi i. Na podstawie prawa przepływu można zapisać że:

\begin{equation} \label{eq:natpol1}
	\oint\limits_C H dl = i
\end{equation}

Natężenie pola H jest stałe na całym obwodzie okręgu g (Pole jest symetryczne)

\begin{equation} \label{eq:natpol2}
	\oint\limits_C H dl = H \int\limits_C dl = H 2 \pi r
\end{equation}

Porównując stronami równania (\ref{eq:natpol1}) i (\ref{eq:natpol2}) natężenie pola w punkcie K można obliczyć ze wzoru:

\begin{equation} \label{eq:natpol3}
	H = \frac{i}{2 \pi r}
\end{equation}

Jednostka natężenia pola magnetycznego to amper na metr [$ \frac{A}{m} $], a jego definicja to:
	   
\begin{defn}
	Amper na metr jest to pole magnetyczne, jakie występuje na powierzchni bocznej walca kołowego o obwodzie 1 m, stycznie do powierzchni bocznej tego walca i prostopadle do jego tworzącej, gdy przez znajdujący się w osi tego walca przewód prostoliniowy nieskończenie długi o przekroju kołowym znikomo małym płynie nie zmieniający się prąd 1 A \cite{kaluszko}.
\end{defn}






