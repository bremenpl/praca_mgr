\section{Elektromagnetyzm}

Aby zrozumieć zasadę działania większości silników elektrycznych, należy najpierw zaznajomić się z podstawowymi jednostkami występującymi w elektromagnetyzmie, dzięki którym silniki działają. Rozdział został opracowany na podstawie źródła \cite{jaszczuk}

\subsection{Natężenie pola magnetycznego}

Pole magnetyczne wytwarza się wokół przewodnika przez który płynie prąd elektryczny. Pole to charakteryzują wektory natężenia, które w każdym punkcie są styczne do okręgów linii sił pola. Środek każdego z okręgów jest zbieżny z osią geometryczną przewodu (przewodnika). Stosując regułę prawej dłoni (Rysunek \ref{grafiki/right_hand_rule.pdf}), palce wskazują zwrot wektorów, a kciuk zwrot i kierunek płynącego prądu.

\insertImg{grafiki/right_hand_rule.pdf}
	   {Reguła prawej dłoni - sposób wyznaczania zwrotu linii sił pola magnetycznego wokół przewodnika w którym jest prąd elektryczny}
	   {wikiRegPrawDlon}
	   
Do wyznaczenia natężenia pola magnetycznego w dowolnym punkcie na zewnątrz przewodu należy posłużyć się rysunkiem \ref{grafiki/zwrot_wektora_nat_pola.pdf}

\insertImg{grafiki/zwrot_wektora_nat_pola.pdf}
	   {Zwrot wektora natężenia pola magnetycznego w przestrzeni wokół przewodu przez który płynie prąd elektryczny (Zwrot za płaszczyznę rysunku od strony obserwatora)}
	   {oprWlasne}
	   
\begin{easylist}
	& K: punkt w przestrzeni wokół przewodnika,
	& g: Okrąg na którym leży punkt K,
	& r: promień okręgu g,
	& H: wektor natężenia pola.
	\\
\end{easylist} 

W przypadku linii pola, która przechodzi przez punkt K, przepływ przez powierzchnię o brzegu g równa się prądowi i. Na podstawie prawa przepływu można zapisać że:

\begin{equation} \label{eq:natpol1}
	\oint\limits_C H dl = i
\end{equation}

Natężenie pola H jest stałe na całym obwodzie okręgu g (Pole jest symetryczne)

\begin{equation} \label{eq:natpol2}
	\oint\limits_C H dl = H \int\limits_C dl = H 2 \pi r
\end{equation}

Porównując stronami równania (\ref{eq:natpol1}) i (\ref{eq:natpol2}) natężenie pola w punkcie K można obliczyć ze wzoru:

\begin{equation} \label{eq:natpol3}
	H = \frac{i}{2 \pi r}
\end{equation}

Jednostka natężenia pola magnetycznego to amper na metr [$ \frac{A}{m} $], a jego definicja to:
	   
\begin{defn}
	Amper na metr jest to pole magnetyczne, jakie występuje na powierzchni bocznej walca kołowego o obwodzie 1 m, stycznie do powierzchni bocznej tego walca i prostopadle do jego tworzącej, gdy przez znajdujący się w osi tego walca przewód prostoliniowy nieskończenie długi o przekroju kołowym znikomo małym płynie nie zmieniający się prąd 1 A \cite{kaluszko}.
\end{defn}

\subsection{Indukcja magnetyczna}

Do opisu pola magnetycznego po za natężeniem pola magnetycznego H, potrzebna jest jeszcze jedna wielkość wektorowa- Indukcja magnetyczna B. W środowisku izotropowym\footnote{Jednakowe własności fizyczne we wszystkich kierunkach.} H i B są do siebie proporcjonalne. Współczynnik proporcjonalności to przenikalność magnetyczna środowiska $ \mu $. Indukcję magnetyczną można zapisać równaniem:

\begin{equation} \label{eq:indmag1}
	B = \mu H
\end{equation}

Wielkość $ \mu $, która określa własności magnetyczne środowiska przedstawia się zależnością:

\begin{equation} \label{eq:indmag2}
	\mu = \mu_0 \mu_r
\end{equation}

\begin{easylist}
	& $ \mu_0 $: przenikalność magnetyczna próżni, która wynosi:
	
	\begin{equation} \label{eq:indmag3}
		\mu_0 = 4 \pi 10^{-7} \frac{H}{m}
	\end{equation}
	
	& $ \mu_r $: względna przenikalność magnetyczna środowiska.
	\\
\end{easylist} 

Jednostką przenikalności magnetycznej jest Henr na metr [$ \frac{H}{m} $], a jej definicja to:

\begin{defn}
	Henr na metr jest to przenikalność magnetyczna bezwzględna środowiska izotropowego w którym polu magnetycznemu 1 [$ \frac{A}{m} $] odpowiada indukcja magnetyczna 1 T (tesla) \cite{kaluszko}.
	
	\begin{equation} \label{eq:indmag4}
		1 \frac{H}{m} = \frac{1 T}{1 \frac{A}{m}}
	\end{equation}
\end{defn}

Wyrażając to w innych jednostkach:

\begin{equation} \label{eq:indmag4}
	1 \frac{H}{m} = 1 \frac{N}{A^2}
\end{equation}

Równanie w którym występuje jednostka siły jest szczególnie przydatne w sytuacjach, w których analizowane są siły wytwarzane przez pole magnetyczne towarzyszące przepływowi prądu (np. Aplikacje z udziałem silników elektrycznych).

\subsubsection{Materiały ferromagnetyczne} 

Materiały ferromagnetyczne dzieli się na miękkie i twarde magnetycznie. W maszynach elektrycznych stosowane są obwody magnetyczne, które wykonane są z materiałów magnetycznie miękkich. Nie dotyczy to magnesów trwałych stosowanych w niektórych maszynach, które są wykonane z materiałów magnetycznie twardych. Magnetycznie miękkie materiały są łatwe do magnesowania przy użyciu zewnętrznego pola magnetycznego. Po zaniku tego pola magnesowanie także zanika. Właściwości magnetyczne ferromagnetyków określają krzywe magnesowania. Wyróżniamy pierwotną krzywą magnesowania i krzywą zamkniętą pełnego przemagnesowania (pętla histerezy magnetycznej \ref{grafiki/petla_histerezy_mag.png}).

\insertImgSetSize{grafiki/petla_histerezy_mag.png}	
		{80}   
	   	{Krzywa magnesowania przykładowego materiału twardego magnetycznie. Przy natężeniu koercji indukcja magnetyczna jest równa zeru.}
	   	{oprWlasne}
	   
Krzywą magnesowania pierwotną, otrzymuje się dla ferromagnetyka magnesowanego po raz pierwszy od stanu $ H = 0 $ i $ B = 0 $ przy monotonicznie rosnącym natężeniu pola H, aż do nasycenia materiału magnetycznego. Przy przemagnesowaniu ferromagnetyka od $ -H_{max} $ do $ +H_{max} $ i odwrotnie od $ +H_{max} $ do $ -H_{max} $ otrzymujemy symetryczną krzywą zamkniętą, inaczej pętlę histerezy magnesowania. Kształt pętli zależy od wielu czynników, między innymi od składu ferromagnetyka i sposobu jego obróbki. \\

Jednostką indukcji magnetycznej jest tesla [T]. Definicja tej jednostki jest następująca:

\begin{defn}
	Tesla jest to indukcja magnetyczna pola magnetycznego równomiernego, przy której na przekrój poprzeczny $ 1 m^2 $ przypada strumień magnetyczny $ 1 Wb $ (weber) \cite{kaluszko}.
\end{defn}

\begin{equation} \label{eq:indmag5}
	1 T = \frac{1 Wb}{1 m^2} = \frac{1 kg}{1 s^2 \cdot 1 A}
\end{equation}

\subsection{Strumień indukcji magnetycznej}

Strumień magnetyczny, który przepływa przez powierzchnię S, definiowany jest jako iloczyn skalarny wektora indukcji magnetycznej i wektora normalnego do powierzchni S. Jeśli pole magnetyczne $ B = const $ i przechodzi przez płaszczyznę S prostopadłą do lini pola, to iloczyn indukcji magnetycznej B i powierzchni pola S to strumień indukcji magnetycznej (rys. \ref{grafiki/strumien_ind_prost.jpg}). W magnetyzmie strumień indukcji jest odpowiednikiem natężenia prądu w elektryczności. 

\insertImg{grafiki/strumien_ind_prost.jpg}
	   {Strumień indukcji magnetycznej wywołany równomiernym polem magnetycznych o indukcji B prostopadłym do płaszczyzny S}
	   {jaszczuk}
	   
Wyraża się go wzorem:

\begin{equation} \label{eq:strindmag1}
	\Phi = B S
\end{equation}

W przypadku kiedy płaszczyzna S nie jest prostopadła do linii sił pola magnetycznego, to należy wziąć pod uwagę wartość $ \alpha $, która wyraża kąt pochylenia płaszczyzny S do linii pola:

\begin{equation} \label{eq:strindmag2}
	\Phi = B S \cos{\alpha}
\end{equation}

Całkowity strumień indukcji przenikający powierzchnię S będzie całką powierzchniową liczoną po całej powierzchni S.

\insertImg{grafiki/strumien_ind_dow.jpg}
	   {Strumień indukcji magnetycznej przepływający przez dowolną powierzchnię}
	   {jaszczuk}
	   
\begin{equation} \label{eq:strindmag3}
	\Phi = \int\limits_S B dS \cos{\alpha}
\end{equation}
	   
Jednostką strumienia magnetycznego jest weber $ [Wb] $.

\begin{defn}
	Weber jest to strumień magnetyczny, który malejąc jednostajnie do zera w czasie 1 s indukuje siłę elektromotoryczną 1 V w obejmującym ten strumień magnetyczny obwodzie zamkniętym jednozwojowym wykonanym z przewodu o przekroju kołowym znikomo małym \cite{kaluszko}.
\end{defn}   

\begin{equation} \label{eq:strindmag4}
	1 Wb = 1 V \cdot 1 s = \frac{1 kg \cdot 1 m^2}{1 s^2 \cdot 1 A}
\end{equation}

\subsection{Siła elektrodynamiczna}

W przestrzeni wokół przewodnika z prądem wytwarzane jest pole magnetyczne. Jeśli umieścić w tej przestrzeni drugi przewodnik z prądem, który wytwarza własne pole magnetyczne to przewodniki te będą się przyciągały lub odpychały. Siły z którymi oddziałują na siebie oba przewody nazywane są siłami elektrodynamicznymi. Na rysunku \ref{grafiki/sily_elek_przewodniki_z_pradem.jpg} przedstawiono dwa nieskończenie długie, równoległe do siebie przewody z prądami $ i_1 $ i $ i_2 $, które znajdują się w środowisku jednorodnym. Linie pola magnetycznego wytworzonego przez prąd $ i_1 $ to okręgi o środkach na osi przewodu pierwszego i są prostopadłe do osi obu przewodów. Siła działająca na odcinek o długości l przewodu drugiego to:

\begin{equation} \label{eq:silapola1}
	F = B_1 i_2 l
\end{equation}

$ B_1 $ to indukcja magnetyczna pola wytworzonego przez prąd $ i_1 $ (wzór \ref{eq:natpol3}).

\begin{equation} \label{eq:silapola2}
	B_1 = \mu_0 H_1 = \mu_0 \frac{i_1}{2 \pi a}
\end{equation} 

$ a $ to odległość analizowanych przewodów. 

\insertImgSetSize{grafiki/sily_elek_przewodniki_z_pradem.jpg}
		{80}
	   	{Siły działające na równoległe przewodniki z prądem}
	   	{jaszczuk}
	   	
Po podstawieniu wzorów \ref{eq:silapola1} i \ref{eq:silapola2} otrzymuje się:

\begin{equation} \label{eq:silapola3}
	F = B_1 i_2 l = \mu_0 \frac{i_1}{2 \pi a} i_2 l = \mu_0 \frac{i_1 i_2 l}{2 \pi a}	
\end{equation}
	
W przypadku rysunku \ref{grafiki/sily_elek_przewodniki_z_pradem.jpg} zwroty prądów są zgodne, więc przewodniki będą się przyciągać. Jeśli były by przeciwne- odpychały by się. 
Jednostką natężenia prądu elektrycznego jest amper $ [A] $.

\begin{defn}
	Amper jest to prąd elektryczny nie zmieniający się, który płynąc w dwóch róznoległych prostoliniowych, nieskończenie długich przewodach o przekroju kołowym znikomo małym, umieszczonych w próżni w odległości 1 m od siebie wytwarza między tymi przewodami siłę $ 2 \cdot 10^{-7} N $ na każdy metr długości \cite{kaluszko}.
\end{defn}  
	
Wartość siły elektrodynamicznej działającej na przewodnik z prądem, który jest umieszczony w polu magnetycznym wytwarzanym przez magnes trwały można obliczyć ze wzoru:

\begin{equation} \label{eq:silapola4}
	F = B i l
\end{equation}

\begin{easylist}
	& B: indukcja wytwarzana przez magnes trwały,
	& i: natężenia prądu w przewodniku,
	& l: długość przewodnika w zasięgu indukcji B.
	\\
\end{easylist} 

\insertImg{grafiki/sily_elek_magnes.jpg}
	   {Siła elektrodynamiczna działająca na przewodnik z prądem umieszczony w stałym polu magnetycznym magnesu trwałego}
	   {jaszczuk}
	   
Zjawisko z rysunku \ref{grafiki/sily_elek_magnes.jpg} jest wykorzystywane do budowy elektrodynamicznych silników prądu stałego.
Kierunek działania siły elektrodynamicznej można wyznaczyć posługując się regułą trzech palców lewej dłoni (Reguła Flemminga).
	   
\subsection{Samoindukcja} 
	
Indukowana w obwodzie elektrycznym siła elektromotoryczna (SEM) jest równa prędkości, z jakązmienia się strumień magnetyczny przechodzący przez ten obwód. Powstawanie indukowanej siły elektromotorycznej zachodzi pod wpływem względnego ruchu źródła pola magnetycznego i obwodu.

\begin{equation} \label{eq:samoind1}
	\varepsilon = - \frac{d \Phi}{d t}
\end{equation}

\begin{easylist}
	& $ \varepsilon $: indukowane napięcie,
	& $ \Phi $: strumień indukcji magnetycznej B,
	& t: czas.
	\\
\end{easylist} 
	
Minus we wzorze bierze się z zasady zachowania energii- oznacza że SEM jest skierowana tak, aby przeciwdziałać przyczynie jej powstawania (Zasada przekory Lenza).

Indukcja własna (samoindukcja) jest szczególnym przypadkiem zjawiska indukcji elektromagnetycznej i występuje, gdy SEM wytwarzana jest w tym samym obwodzie, w którym jest prąd. Zachodzi, gry prąd zmienia swoją wartość, przez co zmienia się pole magnetyczne przez niego wytwarzane. Powstająca siła elektromotoryczna samoindukcji przeciwstawia się natężeniu prądu płynącego dotychczas. Samoindukcja spowalnia narastanie prądu i opadanie prądu w obwodzie, odpowiednio przy załączeniu i odłączeniu zasilania od obwodu. SEM samoindukcji $ \varepsilon_L $ opisuje wzór:

\begin{equation} \label{eq:samoind2}
	\varepsilon_L = - L \frac{d i(t)}{d t}
\end{equation}

Wartość SEM samoindukcji zależy od prędkości zmian prądu w obwodzie i od indukcyjności obwodu L. Gdy w otoczeniu obwodu nie ma obiektów ferromagnetycznych, to przenikalność magnetyczna ośrodka $ \mu_r = 1 $, a indukcyjność w równaniu \ref{eq:samoind2} jest stała i zależy tylko od geometrii obwodu. Indukcyjność obwodu określa jego zdolność do wytwarzania strumienia pola magnetycznego $ \Phi_L $, powstającego w wyniku przepływu przez obwód prądu elektrycznego i:

\begin{equation} \label{eq:samoind3}
	\Phi_L = L i
\end{equation}

Jednostką indukcyjności jest henr $ [H] $.
\begin{defn}
	Henr jest to indukcyjność obwodu, w którym indukuje się siła elektromotoryczna 1 V, gdy prąd elektryczny płynący w tym obwodzie zmienia się jednostajnie o 1 A w czasie 1 s \cite{kaluszko}.
\end{defn}  

\begin{equation} \label{eq:samoind3}
	1 H = \frac{1 kg \cdot 1 m^2}{1 s^2 \cdot 1 A} = \frac{1 Wb}{1 A}
\end{equation}

W obecności ferromagnetyków w otoczeniu przewodnika z prądem zmiana natężenia prądu, która powoduje zmianę natężenia pola magnetycznego, powoduje z kolei zmianę przenikalności magnetycznej. Oznacza to, że indukcyjność obwodu elektrycznego z prądem jest wówczas funkcją natężenia prądu płynącego w tym obwodzie.

\begin{equation} \label{eq:samoind4}
	\Phi_L = L(i) i
\end{equation}

Zależność siły elektromotorycznej samoindukcji $ \varepsilon_L $ od zmian natężenia prądu przyjmuje postać:

\begin{equation} \label{eq:samoind5}
	\varepsilon_L = - \frac{d \Phi_L}{d t} = - (L \frac{di}{dt} + i \frac{dL}{dt})
\end{equation}

\subsection{Reluktancja}
	
Dla obwodu magnetycznego obowiązuje podobna zależność jak dla obwodu elektrycznego i prawa Ohma: napięcie magnetyczne $ U_\mu $ wzdłuż odcinka obwodu magnetycznego równa się iloczynowi oporu magnetycznego $ R_\mu $ i strumienia magnetycznego $ \Phi $ w tym odcinku i jest spowodowane przepływem tego strumienia. Napięcie magnetyczne dla całego obwodu jest równe sumie napięć dla jego fragmentów, jeśli płynie przez nie ten sam strumień magnetyczny:

\begin{equation} \label{eq:reluk1}
	U_\mu = \Phi R_\mu
\end{equation}

Reluktancja jest odpowiednikiem magnetycznym rezystancji, czyli jest to inaczej opór magnetyczny.

\begin{equation} \label{eq:reluk2}
	R_\mu = \frac{l}{\mu S} = \frac{l}{\mu_r \mu_0 S}
\end{equation}

\begin{easylist}
	& $ l $: długość obwodu $ [m] $,
	& $ \mu $: przenikalność magnetyczna materiału $ [\frac{H}{m}] $,
	& $ \mu_r $: względna przenikalność magnetyczna materiału (bezwymiarowa),
	& $ \mu_0 $: przenikalność magnetyczna próżni $ [\frac{H}{m}] $,
	& $ S $: przekrój poprzeczny obwodu $ [m] $.
	\\
\end{easylist} 
	 	
\clearpage
	   



