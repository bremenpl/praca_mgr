\section{Oprogramowanie}
\label{s:oprogramowanie}

Kod wykonywany w mikroprocesorze sterującym został napisany w języku c i w nielicznych częściach w assemblerze. Jądrem całej architektury jest system operacyjny czasu rzeczywistego FreeRTOS™. Wszystkie procedury wysokiego poziomu wykonywane w kodzie są umieszczone w wątkach o odpowiednich priorytetach, które porozumiewają się między sobą za pomocą kolejek, semaforów oraz obszarów pamięci współdzielonej. Zadania najniższego poziomu są wykonywane w procedurach obsługi przerwań, np. komutacja lub wkładanie przychodzących ramek CAN do kolejki. \\

Rysunek \ref{grafiki/Soft_architecture.pdf} przedstawia poglądowy schemat blokowy architektury systemu. Pomarańczowymi blokami zostały oznaczone procedury wykonywane na poziomie sprzętowym, lub programowym na warstwie znajdującej się niżej od wątków systemu FreeRTOS™. Kolor zielony oznacza wątki/ procesy w systemie FreeRTOS™. Strzałki z paskami symbolizują komunikaty przychodzące do kontrolera ze świata zewnętrznego, a bez nich komunikaty wewnątrz systemowe. Dokładniejszy opis poszczególnych bloków znajduje się w korespondujących z nimi sekcjach.

\insertImgSetSize{grafiki/peryferia.png}
	{110}
	{Widok wstępnie skonfigurowanych peryferiów w programie mxCube firmy ST Microelectronics}
	{oprWlasne}
	
Do wygenerowania kodu zawierającego wstępną konfigurację peryferiów służy program mxCube firmy ST Microelectronics.
	
\insertImgSetSize{grafiki/Soft_architecture.pdf}
	{220}
	{Schemat blokowy architektury oprogramowania sterownika}
	{oprWlasne}
	
\clearpage

Jego możliwości konfiguracyjne są bardzo podstawowe, ale pozwala on w graficzny sposób zobrazować używane moduły i wyprowadzenia (rys. \ref{grafiki/peryferia.png}). Dla bardziej skomplikowanych algorytmów przed implementacją zostały wykonane symulacje w programie MATLAB, który posłużył także do zobrazowania danych.

\subsection{Komunikacja}

Gotowe do obsługi programowej w urządzeniu są dwie magistrale szeregowe (patrz \ref{sss:hardware_komunikacja}). W tej sekcji znajduje się opis zastosowania każdej z nich na obecnym poziomie rozwoju.

\subsubsection{Magistrala CAN}
\label{ss:hwcan}

Magistrala pracuje w protokole CAN 2.0B (rozszerzona ramka: 29 bitowy identyfikator i 64 bitowe pole danych) z prędkością 1 Mbit/s. Używany protokół to TML CAN, który jest używany przed dotychczas używane w maszynach sterowniki firmy Technosoft. Jak zostało wcześniej wspomniane, projektowane urządzenie ma za zadanie zastąpić w maszynie dwa dotychczasowe, dlatego z punktu widzenia sieci są to dwa kontrolery (rys. \ref{grafiki/can_nodes.pdf}).

\insertImgSetSize{grafiki/can_nodes.pdf}
	{60}
	{Topologia sieci CAN. Sterownik przedstawia się jako dwa osobne urządzenia}
	{st}
	
W rzeczywistości kontroler jest tylko jeden. Mikrokontroler STM32F07VTG6 posiada dwie sprzętowe kolejki typu FIFO ({\it First In First Out}) do odbierania ramek CAN, które są współdzielone przez moduły CAN1 i CAN2. Urządzenie korzysta tylko z jednego peryferia, więc obie kolejki mogą być zastosowane przez nie. Filtry ramek przychodzących są skonfigurowane w taki sposób, aby przepuszczać tylko te ramki, które posiadają w swoim identyfikatorze numer osi Z lub R (odpowiednio 5 i 6). Po odebraniu wiadomości dla osi Z, ramka trafia do pierwszej kolejki FIFO, jeśli wiadomość jest dla osi R, to do drugiej. Taka architektura zapewnia bardzo dobre odseparowanie wirtualnych modułów odbiorczych sterowników już na poziomie sprzętowym. Bufory FIFO mają jednak ograniczoną pojemność-- mieszczą tylko 3 ramki. Z tego względu wiadomości do nich wpadające muszą być jak najszybciej odczytane i ulokowane w programowej kolejce o rozmiarze 32 ramek, w której mogą oczekiwać na interpretację przez dłuższy okres. \\

Dostęp do zmiennych (obszarów pamięci) w sterownikach Technosoft'u odbywa się poprzez zapis do i odczyt z rejestrów. Architektura konstruowanego kontrolera nie wspiera w natywny sposób takiego programowania, dlatego w celu poprawnej pracy potrzebna jest dodatkowa warstwa (rys. \ref{grafiki/tml_can_trans.pdf}).

\insertImgSetSize{grafiki/tml_can_trans.pdf}
	{35}
	{Sposób tłumaczenia ramek TML CAN w kontrolerze. Przykładowy zapis prędkości ruchu w osi Z}
	{oprWlasne}
	
W warstwie tłumaczenia ramek TML dokonywanych jest wiele różnych operacji. W zależności od komunikatu może to być zapis do zmiennej lub np. wywołanie funkcji. W przykładzie zaprezentowanym na rysunku \ref{grafiki/tml_can_trans.pdf} zewnętrzne urządzenia wysyła do kontrolera wiadomość TML, która po przeskalowaniu ustawia prędkość ruchu w osi Z.

\subsubsection{Magistrala RS232}
\label{ss:hwrs232}

Docelowo interfejs RS232 ma umożliwiać alternatywną metodę komunikacji \linebreak z urządzeniem w celu programowania i sterowania. Obecnie jednak jest to kanał typowo serwisowy. Korzysta z niego moduł logera (rys. \ref{grafiki/Soft_architecture.pdf}) aby wysyłać do aplikacji komputerowej informacje o zdarzeniach w HRF ({\it Human Readable Format}). Typowa ramka tego typu zawiera informacje o czasie wystąpienia zdarzenia, poziomie logowania (np. informacja o priorytecie normalnym lub krytycznym) oraz tekst dowolny (listing \ref{kody/rs232.txt}). 

\insertCode{kody/rs232.txt}
		   {}
		   {Przykładowe logi informujące o wywołaniu funkcji poprzez CAN}
		   {oprWlasne}
		   
Innym trybem logowania jest format CSV ({\it Comma Separated Values}), który pozwala na zbieranie danych w czasie rzeczywistym w celu ich dalszej analizy w programach MATLAB lub Excel. Obsługa modułu logera nie pochłania dużej ilości czasu procesora ze względu na to, że wątek ściągający kolejne logi z kolejki oraz kanał DMA, przez które są one wysyłane mają najniższe priorytety-- procedury wykonują się tylko wtedy, kiedy pozostałe procesy są uśpione.



\clearpage









\subsection{Realizacja zadań}

Wysunięcie się cylindra silnika VCM o odległość $ x $, zwrócenie wartości podciśnienia w głowicy czy obrót komponentu o 90\degree -- wszystko to jest przykładam wywołania funkcji, czy też realizacją zadania. Każda funkcja ma na wykonanie się określoną ilość czasu i jest zbudowana na zasadzie skryptu, który czeka na zajście konkretnych wydarzeń ({\it Events}), interpretuje je i wykonuje kolejne procedury w zależności od ich wyniku. W przypadku nie wykonania się którejkolwiek części skryptu w zadanym czasie, skrypt jest anulowany i do urządzenia wydającego komendę zostaje zwrócony kod błędu. System interpretuje obecnie 9 zdarzeń (listing \ref{kody/sys_triggers.h}).

\insertCode{kody/sys_triggers.h}
		   {C}
		   {Obiekt typu {\it enum} prezentujący obsługiwane w sterowniku wydarzenia}
		   {mechsyscode}

Zdarzenia mogą zachodzić sekwencyjnie lub równolegle. Skrypt może oczekiwać na zajście kilku skryptów (operatory {\it AND} i {\it OR}), gdzie każdy z nich także posiada ustalony czas na wykonanie się. \\

Rysunek \ref{grafiki/uml_homing.pdf} przedstawia przykładową procedurę (homing osi Z) w języku UML ({\it Unified Modeling Language}) wykonywaną przez system. W pierwszej kolejności zewnętrzne urządzenie (Kontroler systemowy automatu Pick and Place) wysyła poprzez magistralę CAN wywołanie funkcji Homing'u. Ramka trafia do kolejki odbiorczej CAN. Po interpretacji zostaje wybrany odpowiedni skrypt ({\it JobDoHoming}) i rozpoczyna się jego wykonywanie. \\

Pierwszym etapem działania skryptu jest zamknięcie zaworu podciśnienia. Następnie sprawdza na którym polu paska referencyjnego znajduje się część ruchoma głowicy (sekcja \ref{ss:encopt}). Jeśli jest to pole czarne (cylinder silnika VCM wsunięty), następuje wejście w pierwsze rozwidlenie w skrypcie (Black strip). Procedura homing'u służy do kalibracji pozycji. Zanim zostanie wykonana, absolutna pozycja cylindra nie jest znana. Po włączeniu zasilania przyjmuje się że pozycja w której aktualnie znajduje się cylinder to zero. Wiedząc tylko że cylinder jest w polu czarnym paska, następuje wysuw cylindra według określonej trajektorii (sekcja \ref{sss:trajgen}). Od razu po tym, funkcja oczekuje na zejście z czarnej strefy paska na białą (wiadomość {\it EXTI} na diagramie). Kiedy to nastąpi, zostaje ustalona absolutna pozycja cylindra. Od tej chwili można swobodnie sterować wysuwem głowicy. Na końcu cylinder wraca do ustalonej pozycji startowej.

\insertImgSetSize{grafiki/uml_homing.pdf}
	{130}
	{Diagram interakcji prezentujący wykonanie przez system procedury bazowania ({\it homing}) w osi Z}
	{oprWlasne}

Jeśli z jakiegoś powodu podczas uruchomienia procedury homing'u cylinder nie jest wsunięty (np. rozciągnięta lub uszkodzona sprężyna) i głowica znajduje się nad polem białym, najpierw następuje jej wsunięcie, aby homing zawsze następował \linebreak w ten sam sposób (zejście z pola czarnego na białe). Potwierdzenie wykonania zadania jest wysyłane do urządzenia wywołującego funkcję, w tym wypadku do kontrolera systemu. \\

Wszystkie zadania obecnie obsługiwane przez sterownik są wykonywane na prezentowanej na rys. \ref{grafiki/uml_homing.pdf} zasadzie (np. podniesienie komponentu, położenie go lub obrócenie). 






\subsection{Algorytmy enkodera magnetycznego}

W odróżnieniu do enkodera liniowego, dane odczytywane ze skonstruowanego modułu enkodera magnetycznego bazującego na układzie AS5048A (rys. \ref{grafiki/AS5048A_diagram.eps}) są obarczone błędami związanymi z nieliniowością układu i typem enkodera. Bez zastosowania żadnych algorytmów uśredniających odczytaną pozycję kątową, dokładność systemu była by marna, a sam układ napędowy działał by niestabilnie. \\

Aby umożliwić pracę enkodera z rozdzielczością do 0.05\degree{} w sterowniku zostały zaimplementowane następujące algorytmy:

\begin{easylist}
	& Uśrednianie bazujące na średniej arytmetycznej,
	& Filtr Kalmana (opracowany na podstawie \cite{kalman} i \cite{forbot}),
	& Filtr dolnoprzepustowy jednobiegunowy, rekursywny (opracowany na podstawie \cite{smith}).
	\\
\end{easylist} 

\subsubsection{Uśrednianie}

Pierwszy z wymienionych algorytmów wykonywany jest zawsze. Pozostałe dwa natomiast są uruchamiane naprzemiennie, w zależności od konfiguracji użytkownika. Uśrednianie działa na prostej zasadzie-- Wykonywanych jest kilka odczytów pozycji z enkodera (ilość odczytów zależy od konfiguracji, najczęściej cztery), z których następnie obliczana jest średnia arytmetyczna (wzór \ref{eq:alg1}). W kolejnym kroku odczyt jest normalizowany (wzór \ref{eq:alg2})-- mechanizm jest wymagany kiedy pozycja waha się między 359-tym a zerowym stopniem.

\begin{equation} \label{eq:alg1}
	pos = \frac{a_0 + a_1 + a_2 + ... + a_n}{n} 
\end{equation}

\begin{equation} \label{eq:alg2}
pos = \left\{
  \begin{array}{ll}
    pos + 360 & : pos < 0 \\
    pos - 360 & : pos > 360
  \end{array}
\right.
\end{equation}

\begin{easylist}
	& $ pos $ : absolutna pozycja enkodera [deg],
	& $ n $ : ilość próbek do uśrednienia.
	\\
\end{easylist} 

Zbyt duża ilość odczytów nie jest opłacalna, gdyż błąd odczytu przestaje znacząco maleć (wzór \ref{eq:alg3}).

\begin{equation} \label{eq:alg3}
	x = \frac{1}{\sqrt{n}}
\end{equation}

\begin{easylist}
	& $ x $ : szum enkodera (max. 0.06\degree{}),
	& $ n $ : ilość próbek do uśrednienia.
\end{easylist} 

\subsubsection{Filtr Kalmana}

Zastosowany algorytm jest metodą filtracji dynamicznej. Posługując się tym narzędziem, można wyznaczyć pomiarowo niedostępne zmienne jedynie na podstawie bieżących wartości wielkości pomiarowo dostępnych oraz znajomości modelu matematycznego łączącego ze sobą obydwie te grupy pomiarów. \\

Proces można przedstawić za pomocą dyskretnego modelu w przestrzeni stanu:

\begin{equation} \label{eq:alg4}
	x(t + 1) = Ax(t) + Bu(t) + v(t)
\end{equation}

\begin{equation} \label{eq:alg5}
	y(t) = Cx(t) + w(t) 
\end{equation}

Gdzie:
\begin{easylist}
	& $ x(t) $ : stan w chwili czasu $ t = 0, 1,... $,
	& $ y(t) $ : wyjście układu,
	& $ A $ : macierz stanu,
	& $ B $ : macierz wejścia,
	& $ C $ : macierz wyjścia,
	& $ v(t) $ : szum procesowy,
	& $ w(t) $ : szum pomiarowy,
\end{easylist} 

\insertImgSetSize{grafiki/przebieg_kf.eps}
	{110}
	{Przebieg czasowy prezentujący pozycję wału silnika krokowego: zadaną, mierzoną przefiltrowaną przez {\it KF} i  mierzoną nieprzefiltrowaną}
	{oprWlasne}

Zastosowany filtr jest dwuwymiarowy.Oznacza to że estymacji są poddawane dwa parametry: pozycja i prędkość. Model stanowy wygląda następująco:

\begin{equation} \label{eq:alg6}
	x = \begin{bmatrix}
       	pos \\ 
       	vel
     	\end{bmatrix}
\end{equation}

\begin{equation} \label{eq:alg7}
	A = \begin{bmatrix}
       	1 & -dt \\ 
       	0 & 1
     	\end{bmatrix}
\end{equation}

\begin{equation} \label{eq:alg8}
	B = \begin{bmatrix}
       	dt \\ 
       	0
     	\end{bmatrix}
\end{equation}

\begin{equation} \label{eq:alg9}
	C = \begin{bmatrix}
       	1 & 0 
     	\end{bmatrix}
\end{equation}

Implementacja dwuwymiarowa zapewnia dokładniejszą estymatę pozycji niż w przypadku kiedy pod uwagę brane było by tylko położenie. Jeszcze lepsze wyniki można by otrzymać stosując estymatę trójwymiarową (dodatkowo dodac parametr przyspieszenia). Niestety operacje na macierzach 3 x 3 pochłaniały za dużo zasobów procesora i z tego rozwiązania w sterowniku zrezygnowano. \\

Rys. \ref{grafiki/przebieg_kf.eps} prezentuje różne pozycje kątowe w funkcji czasu odczytane ze sterownika w trybie CSV (patrz sekcja \ref{ss:hwrs232}). Wstępnie widać że algorytm działa poprawnie, lecz jego zalety i wady można dokładniej zaobserwować na przebiegu z rys. \ref{grafiki/przebieg_kf_zoom.eps}.

\insertImgSetSize{grafiki/przebieg_kf_zoom.eps}
	{110}
	{Przebieg czasowy (powiększenie przy zmianie pozycji) prezentujący pozycję wału silnika krokowego: zadaną, mierzoną przefiltrowaną przez {\it KF} i  mierzoną nieprzefiltrowaną}
	{oprWlasne}
	
Po doświadczalnym dostosowaniu parametrów $ v $ i $ w $ algorytm tłumi zakłócenia bardzo dobrze. Charakterystyczną cechą filtru Kalmana jest jednak to że wraz ze wzrostem poziomu filtracji rośnie także opóźnienie sygnału. Parametr ten diametralnie wpływa na działanie regulatorów pozycji i prędkości (sekcja <dodac sekcje>). Pomimo tego że estymowana pozycja jest bardzo blisko zadanej, to czas potrzebny na jej osiągnięcie (w końcowym odcinku) jest zbyt długi. Z tego powodu został zaimplementowany dodatkowy, szybszy algorytm stosowany w czasowo krytycznych sytuacjach.

\clearpage

\subsubsection{Filtr dolnoprzepustowy}

Drugim algorytmem stosowanym w celu minimalizacji wpływu szumów towarzyszących odczytywanej pozycji kątowej z enkodera magnetycznego jest rekursywny (lub też {\it NOI}-- o nieskończonej odpowiedzi impulsowej), jednobiegunowy filtr dolnoprzepustowy. Zastosowane zostały jedynie dwa współczynniki rekursywne $ a_0 $ i $ b1 $:

\begin{equation} \label{eq:alg10}
	a_0 = 1 - x
\end{equation}

\begin{equation} \label{eq:alg11}
	b_1 = x
\end{equation}

Współczynnik $ x $ wskazuje jakie jest opadanie przebiegu mierzone między kolejnymi próbkami. Wartość ta jest obliczana na podstawie stałej czasowej filtru:

\begin{equation} \label{eq:alg12}
	x = \exp^{-2 \pi f_{CR}}
\end{equation}

$ f_{CR} $ jest częstotliwością odcięcia dobieraną w sterowniku doświadczalnie (obecnie 20 Hz). Kolejne próbki na wyjściu filtra obliczane są według wzoru \ref{eq:alg13}

\begin{equation} \label{eq:alg13}
	output(t) = input(t) a_0 + input(t - 1) b_1 
\end{equation}

\insertImgSetSize{grafiki/przebieg_lowpass_zoom.eps}
	{110}
	{Przebieg czasowy (powiększenie przy zmianie pozycji) prezentujący pozycję wału silnika krokowego: zadaną,   mierzoną nieprzefiltrowaną i mierzoną przefiltrowaną przez filtr dolnoprzepustowy trzeciego rzędu}
	{oprWlasne}
	
Rząd filtru jest ustalany przez użytkownika. Rys. \ref{grafiki/przebieg_lowpass_zoom.eps} prezentuje odczyt pozycji wału silnika krokowego. Algorytm ten działa szybciej od zaimplementowanego filtru Kalmana-- obrót o 90\degree{} w ok. 250 ms przy zastosowaniu KF i ok. 100 ms przy filtrze dolnoprzepustowym trzeciego rzędu. Ponadto, algorytm jest bardzo prosty i kod potrzebny do jego wykonania nie wymaga dużej ilości zasobów mikrokontrolera (listing \ref{kody/low_pass.c}).

\insertCode{kody/low_pass.c}
		   {C}
		   {Kod filtra dolnoprzepustowego wykonywany w pętli sterującej mikrokontrolera w interwale $ dt = 0.001 s $}
		   {mechsyscode}
		   
Na chwilę obecną uśrednianie wraz z  filtrem dolnoprzepustowym są algorytmami dające najlepsze wyniki działania enkodera.

















\subsection{Algorytmy sterowania wysokopoziomowego}





\clearpage














\clearpage