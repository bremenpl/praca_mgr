\subsection{Sterowanie mostka ,,H''}

Omawiany wcześniej mostek ,,H'' (rys. \ref{grafiki/mostek_h_pn.eps}), który służy do dostarczania prądu do uzwojeń może być sterowany na dwa sposoby-- unipolarny i bipolarny. Oba sposoby mają swoje wady oraz zalety i w związku z tym sterownik, który nie potrafi obsłużyć obu tych trybów może być stosowany tylko w ograniczonych aplikacjach. Ponadto, trybów tych nie należy mylić z trybami zasilania silnika krokowego, które zostały omówione w podrozdziale \ref{sss:sterowanie_krokowy}.

\subsubsection{Modulacja bipolarna}

Sterowaniem bipolarnym nazywa się takie sterowanie, w którym dla kolejnych okresów sygnałów sterujących PWM, wartość napięcia na uzwojeniu zmienia się w zakresie od $ V_d $ (napięcia źródła napięciowego zasilającego silnik) do $ -V_d $. Ten sposób wysterowania mostka przedstawia schemat z rysunku \ref{grafiki/sterowanie_bipolarne.png}. Jako obciążenie został przyjęty silnik DC.

\insertImgSetSize{grafiki/sterowanie_bipolarne.png}
	{80}
	{Silnik DC sterowany bipolarnie}
	{ti_motors}
	
Widoczne na schemacie połączenia i zastosowana bramka NOT jest tylko obrazowym odzwierciedleniem sposobu sterowanie implementowanego programowo w sterowniku. Zakładając sytuację bez obciążenia wirnika, wypełnienie impulsu powyżej 50\% wymusi ruch ,,do przodu'', a poniżej 50\%-- ,,do tyłu''. Taka maszyna jest elektrycznym odpowiednikiem mechanicznej przekładni bezstopniowej CVT ({\it Continuously Variable Transmission}) \cite{ti_motors}. W tego typu maszynie pozycję spoczynkową otrzymuje się poprzez zadawanie sygnału o wartości środkowej. \\

Przy sterowaniu bipolarnym, maszyna pracuje częściowo w czwartym kwadrancie charakterystyki (rys. \ref{grafiki/ch_statyczne.eps}). Tak długo, jak przyłożone do silnika średnie napięcie jest tej samej polaryzacji co SEM generowana w uzwojeniu, i jego amplituda jest wyższa od amplitudy tej SEM, wtedy maszyna pracuje w trybie silnikowym. W przypadku kiedy napięcie zasilania i SEM mają tą samą polaryzację, lecz SEM ma wyższą amplitudę to maszyna pracuje w trybie generatorowym. 

\insertImgSetSize{grafiki/modulacja_bipolarna.png}
	{120}
	{Przebiegi napięć w modulacji bipolarnej}
	{namboodiri_wani}
	
Rys. \ref{grafiki/modulacja_bipolarna.png} prezentuje przebiegu towarzyszące modulacji bipolarnej, zastosowanej w do przykładu mostka z rys. \ref{grafiki/sterowanie_bipolarne.png}. Zadany jest sygnał sinusoidalny $ V_m $, a PWM jest taktowany przebiegiem $ V_{cr} $. $ V_{AG} $ i $ V_{BG} $ to napięcia odpowiednio między potencjałami A i B a masą, natomiast $ V_{AB} $ to spadek na silniku. \\

Zaletą sterowania bipolarnego jest to że z sprzętowego punktu widzenia wymaga bardzo niewiele zasobów. W przypadku najprostszego sterowania wymagany jest tylko jeden sygnał PWM. Oznacza to że nawet najtańszy mikroprocesor może posłużyć do zbudowania sterownika do silnika DC. Ważniejszą zaleta jednak jest to, że używając tylko jednego rezystora pomiarowego ($ R_m $ na rys. \ref{grafiki/sterowanie_bipolarne.png}) istnieje możliwość pomiaru prądu płynącego w uzwojeniu, w dowolnym momencie. Daje to możliwość swobodnego doboru sposobu próbkowania prądu i upraszcza układ sterujący. Najpoważniejszą wadą sterowania bipolarnego jest większa pulsacja prądu w stosunku do metod opisanych w dalszej części pracy.

\subsubsection{Sterowanie unipolarne}
\label{sss:moduni}

Zasadę działania sterowania unipolarnego można zaprezentować bazując na rysunku \ref{grafiki/sterowanie_unipolarne.png}. Przy jego pomocy możliwe jest wysterowanie silnika w taki sposób, aby jego wirnik mógł obracać się w obu kierunkach (kwadranty I i III). Aby pracować w pierwszym kwadrancie, tranzystor Q1 jest włączony, a na bramkę Q4 podawany jest sygnał PWM. Kiedy Q4 jest otwarty, prąd płynie od $ V_d $ przez Q1, dalej przez uzwojenie i przez Q4 do masy poprzez rezystor $ R_m $. Kiedy Q4 się zamyka w uzwojeniu dalej jest prąd ze względu na jego indukcyjność. Zwiększający się potencjał w punkcie połączenia drenu Q4 i źródła Q3 sprawia że dioda tranzystora Q3 zaczyna przewodzić. Z racji tego że Q1 jest cały czas otwarty, obwód prądowy zamknie się właśnie przez ten tranzystor, a nie przez obwód zasilania. Aby obracać wirnikiem w drugą stronę, należy otworzyć Q3 i kluczować Q2. W takim wypadku silnik pracuje w trzecim kwadrancie charakterystyki statycznej (negatywna prędkość i moment. Niezależnie od kierunku przepływu prądu w silniku, napięcie na niego podawane zawsze zmienia się w graniczy od $ V_d $ do GND, a nie jak w schemacie obrazującym sterowanie bipolarne (rys. \ref{grafiki/sterowanie_bipolarne.png}) od $ V_d $ do $ -V_d $. Sprawia to że w takim układzie silnik nie może oddawać energii do zasilania, bo prąd jest ,,zamknięty'' w oczku utworzonym z Silnika, Q1 i Q3. 

\insertImgSetSize{grafiki/sterowanie_unipolarne.png}
	{80}
	{Silnik DC sterowany unipolarnie}
	{ti_motors}


Zaletą prezentowanej topologii jest to że wymagany jest tylko jeden sygnał PWM do sterowania pojedynczym mostkiem (oraz linia wejściowa ustalająca kierunek obrotów). Ponadto, w danym momencie kluczowany jest tylko jeden tranzystor, a więc co za tym idzie straty spowodowane przełączaniem są mniejsze niż w sterowaniu bipolarnym. Modulacja unipolarna wywołuje mniejszą amplitudę tętnień prądu niż modulacja bipolarna przy tej samej częstotliwości modulacji \cite{zawirski}. 

\insertImgSetSize{grafiki/modulacja_unipolarna.png}
	{120}
	{Przebiegi napięć w modulacji unipolarnej}
	{namboodiri_wani}

Bez względu na zastosowaną technikę modulacji, do wszystkich silników zasilanych napięciem stałym można użyć mostka typu ,,H'' w celu realizacji tej modulacji. \\

Konstruowany sterownik korzysta z modulacji unipolarnej, ze względu na to że przynosi o wiele więcej korzyści niż bipolarna. Wadą modulacji unipolarnej jest brak możliwości aktywnego hamowania. Dla urządzenia nie jest to jednak żadnym problemem, ponieważ sposób w jaki uzwojenia są sterowane wyklucza takie sytuacje (więcej informacji na ten temat znajduje się w rozdziale \ref{s:oprogramowanie}.




