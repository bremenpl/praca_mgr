\section{Elektromagnetyzm}

Aby zrozumieć zasadę działania większości silników elektrycznych, należy najpierw zaznajomić się z podstawowymi jednostkami występującymi w elektromagnetyzmie, dzięki którym silniki działają. Rozdział został opracowany na podstawie źródła \cite{jaszczuk}

\subsection{Natężenie pola magnetycznego}

Pole magnetyczne wytwarza się wokół przewodnika przez który płynie prąd elektryczny. Pole to charakteryzują wektory natężenia, które w każdym punkcie są styczne do okręgów linii sił pola. Środek każdego z okręgów jest zbieżny z osią geometryczną przewodu (przewodnika). Stosując regułę prawej dłoni (Rysunek \ref{grafiki/right_hand_rule.pdf}), palce wskazują zwrot wektorów, a kciuk zwrot i kierunek płynącego prądu.

\insertImg{grafiki/right_hand_rule.pdf}
	   {Reguła prawej dłoni - sposób wyznaczania zwrotu linii sił pola magnetycznego wokół przewodnika w którym jest prąd elektryczny}
	   {wikiRegPrawDlon}
	   
Do wyznaczenia natężenia pola magnetycznego w dowolnym punkcie na zewnątrz przewodu należy posłużyć się rysunkiem \ref{grafiki/zwrot_wektora_nat_pola.pdf}

\insertImg{grafiki/zwrot_wektora_nat_pola.pdf}
	   {Zwrot wektora natężenia pola magnetycznego w przestrzeni wokół przewodu przez który płynie prąd elektryczny (Zwrot za płaszczyznę rysunku od strony obserwatora)}
	   {oprWlasne}
	   
\begin{easylist}
	& K: punkt w przestrzeni wokół przewodnika,
	& g: Okrąg na którym leży punkt K,
	& r: promień okręgu g,
	& H: wektor natężenia pola.
	\\
\end{easylist} 

W przypadku linii pola, która przechodzi przez punkt K, przepływ przez powierzchnię o brzegu g równa się prądowi i. Na podstawie prawa przepływu można zapisać że:

\begin{equation} \label{eq:natpol1}
	\oint\limits_C H dl = i
\end{equation}

Natężenie pola H jest stałe na całym obwodzie okręgu g (Pole jest symetryczne)

\begin{equation} \label{eq:natpol2}
	\oint\limits_C H dl = H \int\limits_C dl = H 2 \pi r
\end{equation}

Porównując stronami równania (\ref{eq:natpol1}) i (\ref{eq:natpol2}) natężenie pola w punkcie K można obliczyć ze wzoru:

\begin{equation} \label{eq:natpol3}
	H = \frac{i}{2 \pi r}
\end{equation}

Jednostka natężenia pola magnetycznego to amper na metr [$ \frac{A}{m} $], a jego definicja to:
	   
\begin{defn}
	Amper na metr jest to pole magnetyczne, jakie występuje na powierzchni bocznej walca kołowego o obwodzie 1 m, stycznie do powierzchni bocznej tego walca i prostopadle do jego tworzącej, gdy przez znajdujący się w osi tego walca przewód prostoliniowy nieskończenie długi o przekroju kołowym znikomo małym płynie nie zmieniający się prąd 1 A \cite{kaluszko}.
\end{defn}

\subsection{Indukcja magnetyczna}

Do opisu pola magnetycznego po za natężeniem pola magnetycznego H, potrzebna jest jeszcze jedna wielkość wektorowa- Indukcja magnetyczna B. W środowisku izotropowym\footnote{Jednakowe własności fizyczne we wszystkich kierunkach.} H i B są do siebie proporcjonalne. Współczynnik proporcjonalności to przenikalność magnetyczna środowiska $ \mu $. Indukcję magnetyczną można zapisać równaniem:

\begin{equation} \label{eq:indmag1}
	B = \mu H
\end{equation}

Wielkość $ \mu $, która określa własności magnetyczne środowiska przedstawia się zależnością:

\begin{equation} \label{eq:indmag2}
	\mu = \mu_0 \mu_r
\end{equation}

\begin{easylist}
	& $ \mu_0 $: przenikalność magnetyczna próżni, która wynosi:
	
	\begin{equation} \label{eq:indmag3}
		\mu_0 = 4 \pi 10^{-7} \frac{H}{m}
	\end{equation}
	
	& $ \mu_r $: względna przenikalność magnetyczna środowiska.
	\\
\end{easylist} 

Jednostką przenikalności magnetycznej jest Henr na metr [$ \frac{H}{m} $], a jej definicja to:

\begin{defn}
	Henr na metr jest to przenikalność magnetyczna bezwzględna środowiska izotropowego w którym polu magnetycznemu 1 [$ \frac{A}{m} $] odpowiada indukcja magnetyczna 1 T (tesla) \cite{kaluszko}.
	
	\begin{equation} \label{eq:indmag4}
		1 \frac{H}{m} = \frac{1 T}{1 \frac{A}{m}}
	\end{equation}
\end{defn}

Wyrażając to w innych jednostkach:

\begin{equation} \label{eq:indmag4}
	1 \frac{H}{m} = 1 \frac{N}{A^2}
\end{equation}

Równanie w którym występuje jednostka siły jest szczególnie przydatne w sytuacjach, w których analizowane są siły wytwarzane przez pole magnetyczne towarzyszące przepływowi prądu (np. Aplikacje z udziałem silników elektrycznych).

\subsubsection{Materiały ferromagnetyczne} 

Materiały ferromagnetyczne dzieli się na miękkie i twarde magnetycznie. W maszynach elektrycznych stosowane są obwody magnetyczne, które wykonane są z materiałów magnetycznie miękkich. Nie dotyczy to magnesów trwałych stosowanych w niektórych maszynach, które są wykonane z materiałów magnetycznie twardych. Magnetycznie miękkie materiały są łatwe do magnesowania przy użyciu zewnętrznego pola magnetycznego. Po zaniku tego pola magnesowanie także zanika. Właściwości magnetyczne ferromagnetyków określają krzywe magnesowania. Wyróżniamy pierwotną krzywą magnesowania i krzywą zamkniętą pełnego przemagnesowania (pętla histerezy magnetycznej \ref{grafiki/petla_histerezy_mag.png}).

\insertImgSetSize{grafiki/petla_histerezy_mag.png}	
		{80}   
	   	{Krzywa magnesowania przykładowego materiału twardego magnetycznie. Przy natężeniu koercji indukcja magnetyczna jest równa zeru.}
	   	{oprWlasne}
	   
Krzywą magnesowania pierwotną, otrzymuje się dla ferromagnetyka magnesowanego po raz pierwszy od stanu $ H = 0 $ i $ B = 0 $ przy monotonicznie rosnącym natężeniu pola H, aż do nasycenia materiału magnetycznego. Przy przemagnesowaniu ferromagnetyka od $ -H_{max} $ do $ +H_{max} $ i odwrotnie od $ +H_{max} $ do $ -H_{max} $ otrzymujemy symetryczną krzywą zamkniętą, inaczej pętlę histerezy magnesowania. Kształt pętli zależy od wielu czynników, między innymi od składu ferromagnetyka i sposobu jego obróbki. \\

Jednostką indukcji magnetycznej jest tesla [T]. Definicja tej jednostki jest następująca:

\begin{defn}
	Tesla jest to indukcja magnetyczna pola magnetycznego równomiernego, przy której na przekrój poprzeczny $ 1 m^2 $ przypada strumień magnetyczny $ 1 Wb $ (weber) \cite{kaluszko}.
\end{defn}

\begin{equation} \label{eq:indmag5}
	1 T = \frac{1 Wb}{1 m^2} = \frac{1 kg}{1 s^2 \cdot 1 A}
\end{equation}






