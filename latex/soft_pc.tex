\subsection{Aplikacja testująca}

Aby umożliwić testowanie sterownika bez konieczności używania kontrolera systemowego maszyny, symulator kontrolera (aplikacja graficzna działająca pod systemem Windows) został wyposażony w dodatkowe komendy obsługujące konstruowaną głowicę układającą. \\

Program pozwala na ustawianie poszczególnych rejestrów (sekcja \ref{ss:hwcan}, rys. \ref{grafiki/tml_can_trans.pdf}) pojedynczo używając przycisków lub zbiorczo korzystając z komend zapisanych w pliku XML ({\it Extensible Markup Language}). Funkcje--skrypty obsługiwane przez system są typowe dla głowicy układającej: \\

\begin{easylist}
	& Bazowanie osi : sstawia osie R i Z w pozycjach początkowych,
	& Ruch : poruszanie się ruchem posuwisto--zwrotnym w osi Z i obrotowym w osi R,
	& Operacje na zaworze : Otwieranie/ zamykanie zaworu podciśnienia,
	& Podnoszenie : Procedura podnoszenia komponentów elektronicznych,
	& Układanie : Procedura układania komponentów elektronicznych,
	& Pomiar wysokości: Skrypt służący do pomiaru wysokości komponentu elektronicznego,
	\\
\end{easylist}  

Dodatkowo korzystając z zakładki {\it Bootloader} (rys. \ref{grafiki/pcapp.png}) użytkownik ma możliwość zaktualizowania oprogramowania sterownika, bez korzystania z zewnętrznego programatora. Mikrokontroler przechodzi w stan Bootloader'a (osobny program) z poziomu którego możliwe jest nadpisanie pamięci flash. \\

\clearpage

\insertImgSetSize{grafiki/pcapp.png}
	{185}
	{Aplikacja (interfejs użytkownika) pozwalająca na zadawanie/ odczytywanie parametrów i wywoływanie funkcji w sterowniku, korzystając z interfejsu CAN}
	{mechsyscode}





