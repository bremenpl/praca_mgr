\subsection{Procedury sterowania wysokopoziomowego}

Sterowanie wyższego poziomu w kontrolerze polega na manipulacji, czy też operacjach na danych, które nie mają jeszcze bezpośredniego wpływu prąd w uzwojeniach czy pozycję kątową wału silnika. Odzwierciedlają one jedynie różne wielkości, które następnie są przekazywane do niższej warstwy oprogramowania, która steruje silnikami (\ref{sss:lowlevel}). \\

Procedury sterowania wysokopoziomowego dla obu rodzajów silników (krokowy i VCM) w większości są identyczne. Odmiennie wyglądają moduły sterowania prądem w uzwojeniach, ze względu na odmienną konstrukcję silników. Rys. \ref{grafiki/pid_vcm.pdf} prezentuje poglądowy diagram sterowania silnika VCM, a rys. \ref{grafiki/pid_stepper.pdf} silnika krokowego.

\clearpage

\insertImgSetSize{grafiki/pid_vcm.pdf}
	{225}
	{Schemat blokowy sterowania wysokopoziomowego silnikiem VCM, zaimplementowanego w kodzie sterownika}
	{oprWlasne}
	
\clearpage
	
\insertImgSetSize{grafiki/pid_stepper.pdf}
	{220}
	{Schemat blokowy sterowania wysokopoziomowego silnikiem krokowym, zaimplementowanego w kodzie sterownika}
	{oprWlasne}
	
\clearpage
	
\subsubsection{Regulatory pozycji i prędkości}
\label{sss:posvelregs}

W celu zapewnienia zadanej pozycji (kątowej w przypadku silnika krokowego i liniowej przy VCM) w sterowniku został zastosowany regulator pozycji typu PID-- proporcjonalno-całkująco-różniczkujący ({\it proportional–integral–derivative}). Jego zasadę działania wyraża wzór \ref{eq:pid1}.

\begin{equation} \label{eq:pid1}
	s(t) = K_p e(t) + K_i \int_0^t e(t)dt + K_d \frac{de(t)}{dt}
\end{equation}

Gdzie:

\begin{easylist}
	& $ s(t) $ : pozycja w danej chwili $ t $ $ [m] $,
	& $ e(t) $ : błąd pozycji $ e(t) = v(t) - v(t - 1) [m] $,
	& $ K_p  $ : współczynnik proporcjonalny $ [m] $,
	& $ K_i  $ : współczynnik całkujący $ [m \cdot s] $,
	& $ K_d  $ : współczynnik różniczkujący $ [\frac{m}{s}] $.
	\\
\end{easylist} 

Implementację algorytmu w kodzie mikrokontrolera przedstawia listing \ref{kody/pid_PosController.c}.

\insertCode{kody/pid_PosController.c}
	{C}
	{Kod regulatora pozycji zaimplementowany w  mikrokontrolerze, wywoływany w interwale $ dt = 0.001 s $}
	{mechsyscode}
	
Funkcja zawiera dodatkowo mechanizm zabezpieczający regulator przed zmagazynowaniem zbyt dużego błędu pozycji, co może spowodować przeregulowanie układu ({\it ,,Anti-Windup''}). Jako parametry wejściowe podawane są aktualna pozycja (odczytana z enkodera) i pozycja zadana. Wyjście regulatora pozycji jest następnie podawane na wejście regulatora prędkości (\ref{grafiki/pid_vcm.pdf}). \\

Regulator prędkości działa w ten sam sposób co regulator pozycji, z tą różnicą że sygnał wyjściowy z regulatora jest wielkością tego samego typu co sygnał wejściowy (w uproszczeniu, bo tylko regulator typu P daje na wyjściu sygnał tej samej wielkości co na wejściu). Suma (lub różnica w zależności od znaku zmiennej zwracanej z funkcji) tych dwóch sygnałów tworzy regulowaną prędkość zadaną. Zastosowanie oddzielnych regulatorów dla pozycji i prędkości pozwala w bardzo precyzyjny sposób konfigurować pracę danego silnika przy pomocy współczynników $ K_p $, $ K_i $ i $ K_d $ ({\it tunning}). Kolejne bloki różnią się w zależności od zastosowanego silnika.

\subsubsection{Generator trajektorii}
\label{sss:trajgen}

W celu większej kontroli nad zadanym położeniem i prędkością, oraz aby usprawnić ogólną pracę silników w procedurze sterowania został dodany generator trajektorii. Moduł ten ma za zadanie doprowadzić element czynny maszyny do zadanej pozycji w optymalny sposób.

\insertImgSetSize{grafiki/traj_trapez.eps}
	{105}
	{Wygenerowana trajektoria w kształcie trapezu}
	{oprWlasne}
	
Sposób optymalny w rozumieniu algorytmu jest to próba uzyskania przebiegu prędkości w zadanym odcinku w kształcie trapezu, co pozwala na płynny start \linebreak i hamowanie (rys. \ref{grafiki/traj_trapez.eps}). W celu oszczędzenia pamięci operacyjnej mikrokontrolera kolejne punkty trajektorii są liczone na bieżąco podczas ,,podążania'' za trajektorią. Do jej wyznaczenia potrzebne są następujące dane:

\begin{easylist}
	& $ v_{max} $ : Maksymalna prędkość do jakiej element czynny może być rozpędzony. Jednostka: $[\frac{m}{s}] $ lub  $ [\frac{deg}{s}] $,
	& $ a_{max} $ : Maksymalne przyspieszenie/ opóźnienie. Jednostka: $[\frac{m}{s^2}] $ lub  $ [\frac{deg}{s^2}] $,
	& $ s $ : Zadana pozycja. Jednostka: $[m] $ lub  $ [deg] $.
	\\
\end{easylist}

Każdy obrót głównej pętli sterującej dostarcza regulatorom kolejnych danych wejściowych. W pierwszej kolejności obliczana jest prędkość. W zależności od tego czy aktualne układ przyspiesza, porusza się ze stałą prędkością lub zwalnia, stosowany jest odpowiedni wzór \ref{eq:pid8}.

\begin{equation} \label{eq:pid8}
	v_{set} =  \left\{
  \begin{array}{ll}
    v_0 - \int_{t_0}^{t} a_{max} dt & : type = decel \\
    v_{max} & : type = const \\
    v_0 + \int_{t_0}^{t} a_{max} dt & : type = accel \\
  \end{array}
\right.
\end{equation}

Gdzie $ t $ oznacza aktualny okres pętli. Korzystając z uzyskanej chwilowej prędkości (liniowej lub kątowej w zależności od silnika) obliczana jest pozycja (\ref{eq:pid9}).

\begin{equation} \label{eq:pid9}
	s_{set} =  \left\{
  \begin{array}{ll}
    \int_{t_0}^{t} v_0 dt - \int_{t_0}^{t} (v_0 - v_{set}(t)) dt & : type = decel \\
    s_{set}(t - 1) + v_{set}(t) dt & : type = const \\
    \int_{t_0}^{t} v_0 dt + \int_{t_0}^{t} (v_{set}(t) - v_0) dt & : type = accel \\
  \end{array}
\right.
\end{equation}

\insertImgSetSize{grafiki/traj_duze_v0.eps}
	{105}
	{Wygenerowana trajektoria dla obiektu poruszającego się ($ v_0 > v_{max} $)}
	{oprWlasne}

W przypadku kiedy kontroler odbiera informację o zmianie pozycji, a aktualnie jest w ruchu ($ v_{set} \neq 0$), następuje aktualizacja trajektorii. Przykład dla którego $ v_{max} $ nowej trajektorii jest mniejsza od aktualnej prezentuje rys. \ref{grafiki/traj_duze_v0.eps}. \\
	
Algorytm radzi sobie także z sytuacją, w której aktualizowana pozycja wymaga zmiany kierunku jazdy (rys. \ref{grafiki/traj_reverse.eps}), lub maksymalna prędkość nie może być osiągnięta ze względu na zbyt małą drogę zapewnioną na rozpęd (lub zbyt małe maksymalne przyspieszenie). Wtedy charakterystyka prędkości przyjmuje kształt trójkąta zamiast trapezu.

\insertImgSetSize{grafiki/traj_reverse.eps}
	{105}
	{Wygenerowana trajektoria dla obiektu poruszającego się w przeciwnym kierunku}
	{oprWlasne}
	
Po osiągnięciu zadanej pozycji regulatory działają już bez udziału generatora trajektorii, próbując zachować odpowiednio docelową pozycję i prędkość dążącą do zera.

\subsubsection{Sterowanie prędkością kątową silnika krokowego}

W modelu sterowania, który został zaimplementowany w oprogramowaniu kontrolera, parametrami decydującymi o aktualnej prędkości kątowej silnika krokowego są:

\begin{easylist}
	& częstotliwość komutacji,
	& tryb pracy (,,bieg'').
	\\
\end{easylist} 

Wartości wymienionych parametrów są automatycznie obliczane w pętli sterowania dla zadanej prędkości obrotowej. W celu   inicjalizacji algorytmu należy podać następujące parametry:

\begin{easylist}
	& ilość pełnych kroków na obrót dla danego silnika,
	& ilość biegów (maksymalna dopuszczalna rozdzielczość pracy mikrokrokowej),
	& maksymalna częstotliwość komutacji $ [Hz] $,
	& maksymalna użyteczna prędkość kątowa dla danego silnika $ [RPM] $.
	\\
\end{easylist} 

Relacja między ilością biegów i rozdzielczością przy pracy mikrokrokowej jest następująca:

\begin{equation} \label{eq:pid7}
	gearsNr = \log_2(resolution) + 1
\end{equation}

W pierwszej kolejności algorytm oblicza wartości progowe prędkości obrotowych dla których następuje ,,automatyczna zmiana biegu''. Wartości te są nieco inne dla przypadku w którym silnik jest rozpędzany i hamowany, dzięki czemu bieg nie jest zmieniany zbyt często przy wartościach skrajnych (działanie bazujące na histerezie, podobnie jak w przerzutniku {\it Shmitta}). 

\insertImgSetSize{grafiki/freq_gear_stepper.eps}
	{110}
	{Wizualizacja algorytmu obliczającego częstotliwość komutacji i bieg dla zadanej prędkości obrotowej wału}
	{oprWlasne}

Dla przypadku rozpędzania wirnika z bezruchu, wirtualna przekładnia rozpoczyna pracę przy największej rozdzielczości mikrokroków. Wraz ze wzrostem prędkości obrotowej częstotliwość komutacji jest zwiększana. Im większa częstotliwość komutacji, tym częściej system musi przerywać pozostałe prace, dlatego maksymalną częstotliwość należy dobrać w taki sposób, aby nie zagłodzić pozostałych procesów w systemie. \\

Po osiągnięciu maksymalnej częstotliwości następuje zmiana biegu, czyli dwukrotne zmniejszenie rozdzielczości mikrokroków. Częstotliwość komutacji także dwukrotnie maleje. Działanie to jest powtarzane aż do osiągnięcia maksymalnej ustalonej prędkości obrotowej (i jednocześnie maksymalnego biegu i częstotliwości komutacji). Algorytm pozwala zoptymalizować pracę silnika poprzez korzystanie \linebreak z największej rozdzielczości tylko wtedy, kiedy jest potrzebna (rozpędzanie wirnika i precyzyjne hamowanie). Rys. \ref{grafiki/freq_gear_stepper.eps} prezentuje w graficzny sposób działanie układu. \\

Można zauważyć, że praca na najwyższym biegu (1 mikrokrok na 1 pełny krok) to tak naprawdę tryb falowy (sekcja \ref{sss:sterowanie_krokowy}, rys. \ref{grafiki/stepper_sterowanie_falowe.eps}). Układ przechodzi do niego w naturalny sposób z trybu mikrokrokowego. W aplikacji takiej jak głowica układająca duży moment obrotowy nie jest wymagany, dlatego nie ma potrzeby implementacji trybu pełnokrokowego na ostatnim biegu. Docelowo jednak, funkcjonalność ta będzie musiała być dodana (np. sterowanie ruchu portalu w osi X i Y).

\subsubsection{Regulator prądu silnika VCM}

Wartość prądu zadanego dla uzwojenia maszyny VCM uzyskuje się poprzez jego estymatę, biorąc pod uwagę właściwości fizyczne układu napędowego (blok {\it VCM CURRENT ESTIMATOR} na schemacie \ref{grafiki/pid_vcm.pdf}). Na wejście estymatora trafia wartość absolutnego (względem obudowy), aktualnego wysuwu cylindra silnika. Zadana wartość wysuwu jest uzyskiwana poprzez pomnożenie stałej czasowej układu przez scałkowaną po czasie prędkość z jaką porusza się korpus (\ref{eq:pid2}).

\begin{equation} \label{eq:pid2}
	stroke(t) = \int_0^t v(t) dt
\end{equation}

Do obliczenia zadanego prądu dla silnika VCM należy się posłużyć modelem fizycznym z rys. \ref{grafiki/sprezyna_vcm.eps}.

\insertImgSetSize{grafiki/sprezyna_vcm.eps}
	{70}
	{Teoretyczny model układu napędowego modułu z silnikiem VCM}
	{oprWlasne}
	
Obiekt $ A $ reprezentuje część ruchomą głowicy, czyli wszystkie komponenty znajdujące się po stronie cylindra (rys. \ref{grafiki/glowica.png}). $ B $ to część nieruchoma, po stronie której znajduje się obudowa silnika. W pierwszej kolejności należy obliczyć działającą na ciało $ A $ siłę sprężyny $ F_s $ (\ref{eq:pid3}):

\begin{equation} \label{eq:pid3}
	F_s = F_{s0} + R_s \cdot S \quad [ N = N + \frac{N}{m} \cdot m ]
\end{equation}

\begin{easylist}
	& $ F_{s0} $ : siła początkowa sprężyny $ [N] $,
	& $ R_s $ : stała sprężyny $ [\frac{N}{m}] $,
	& $ S $ : wysuw cylindra $ [m] $.
	\\
\end{easylist} 

Następnie należy obliczyć siłę $ F_m $ z jaką oddziałuje na ciało grawitacja (\ref{eq:pid4}):

\begin{equation} \label{eq:pid4}
	F_m = m_A \cdot g \cdot \sin \alpha \quad [ N = kg \cdot \frac{m}{s^2} \cdot 1 ]
\end{equation}

\begin{easylist}
	& $ m_A $ : sumaryczna masa części ruchomej głowicy $ [kg] $,
	& $ g $ : przyspieszenie ziemskie $ [\frac{m}{s^2}] $,
	& $ \alpha $ : kąt ustawienia głowicy względem płaszczyzny (normalnie 90\degree{}) $ [deg] $.
\end{easylist} 

\insertImgSetSize{grafiki/vcm_current.eps}
	{100}
	{Przebieg teoretycznego prądu (oś Y) wymaganego w uzwojeniu do wysuwu cylindra na zadaną odległość (oś X)}
	{oprWlasne}

Potem liczona jest wypadkowa siła $ F $ (\ref{eq:pid5}) działająca na ciało $ A $. Im cięższa jest część ruchoma głowicy, tym bardziej jest wysunięty korpus w stanie spoczynku (bezprądowym). 

\begin{equation} \label{eq:pid5}
	F = F_s - F_m \quad [ N = N - N ]
\end{equation}

Ostatnim krokiem jest odczytanie stałej silnika VCM ($ F_c $ -- {\it force constant}) z charakterystyki dostarczonej przez producenta (np. \ref{grafiki/vcm_force_current_plot.eps}) i obliczenie prądu w uzwojeniu dla wymaganego wysuwu cylindra (\ref{eq:pid5}):

\begin{equation} \label{eq:pid6}
	I = \frac{F}{F_c} \quad [ A = \frac{N}{\frac{N}{A}} ]
\end{equation}
	
Na rys. \ref{grafiki/vcm_current.eps} widać wynik symulacji w postaci prądu w funkcji wysuwu cylindra. Obliczenia zostały przeprowadzone przy rzeczywistych wartościach parametrów układu napędowego. Wykres wskazuje na to, że w stanie spoczynku część ruchoma głowicy jest wysunięta na odległość ok. 6.7 mm. Charakterystyka ma charakter liniowy w swojej środkowej części, ze względu na kształt charakterystyki prądowej silnika wprowadzonej do algorytmu jako parametr (\ref{grafiki/vcm_force.eps}). 

\insertImgSetSize{grafiki/vcm_force.eps}
	{100}
	{Charakterystyka prądowa zastosowanego silnika VCM odtworzona w programie używając charakterystyki z karty katalogowej}
	{oprWlasne}

Wyznaczony prąd zadany trafia do właściwego regulatora prądu typu PI. jego realizacja różni się od poprzednich regulatorów tym, że nie ma już członu różniczkującego, który w przypadku regulacji prądu nie dał by wielu korzyści. Ponadto stała czasowa pętli prądowej wynosi $ 100 \mu s $. \\

Wartość prądu jest następnie zamieniana na odpowiednie wypełnienie impulsów w uzwojeniu-- więcej informacji na ten temat jest zawarte w sekcji \ref{sss:lowlevel}.
	
\clearpage

\subsubsection{Regulator prądu silnika krokowego}

Regulatory prądu (dla uzwojeń A i B) w przypadku silnika krokowego próbują przez cały okres pracy maszyny utrzymywać w uzwojeniach taki prąd, aby spełnione było równanie \ref{eq:stepper2}. 

\insertImgSetSize{grafiki/stepper_prady.png}
	{110}
	{Oscylogramy mierzonych spadków napięć na rezystorach pomiarowych w mostkach H silnika krokowego poruszającego się ze stałą zadaną prędkością kątową}
	{oprWlasne}

Prąd maksymalny ($ \theta = 90\degree $) jest przemnażany przez odpowiednie współczynniki (zmienne dla danej pozycji) dla obu uzwojeń (rys. \ref{grafiki/pid_stepper.pdf}). Iloczynami są zadane prądy uzwojeń, które następnie trafiają do wspomnianych regulatorów. Poprawność działania układu potwierdza oscylogram \ref{grafiki/stepper_prady.png}.
	
\clearpage
	
\subsubsection{Sprawdzenie systemu sterowania}

W celu zaprezentowania poprawności działania systemu sterowania wykonano kilka przejazdów (dla obu silników) i zarejestrowano zmienne operacyjne w czasie rzeczywistym używając modułu logger'a ( sekcja \ref{ss:hwrs232}). \\

Pierwszy przebieg (rys. \ref{grafiki/reg_odp_stepper.eps}) prezentuje odpowiedzi regulatorów pozycji oraz prędkości na czterokrotną zmianę pozycji kątowej wału silnika krokowego o 180\degree. Do uzyskania zadanych pozycji posłużył generator trajektorii, który w każdym okresie głównej pętli sterowania oblicza kolejne wartości docelowe dla regulatorów. \\
 
Na wykresie pozycji w funkcji czasu z rys. \ref{grafiki/reg_odp_stepper.eps} można zaobserwować niewielkie opóźnienie sygnału mierzonego względem zadanego. Wynika ono z faktu zastosowania algorytmów filtrujących w module enkodera (sekcja \ref{ss:encalg}). Wykres prędkości jest przykładem trajektorii typu trapezoidalnego, w której nie udało się osiągnąć maksymalnej możliwej prędkości, stąd kształt jest trójkątny.

\insertImgSetSize{grafiki/reg_odp_stepper.eps}
	{105}
	{Przykładowe odpowiedzi regulatorów pozycji i prędkości silnika krokowego, z zastosowaniem generatora trajektorii}
	{oprWlasne}
	
Na rysunku \ref{grafiki/reg_odp_vcm.eps} widać podobne przebiegi dla silnika VCM. Opóźnienie pozycji w funkcji czasu już nie występuje, gdyż zastosowany jest enkoder liniowy (sekcja \ref{ss:encopt}). Zmiany odpowiedzi prędkości są dla tego silnika duże, ze względu na wysoki współczynnik $ K_p $ ustawiony w regulatorze prędkości.

\clearpage
	
\insertImgSetSize{grafiki/reg_odp_vcm.eps}
	{105}
	{Przykładowe odpowiedzi regulatorów pozycji i prędkości silnika liniowego VCM, z zastosowaniem generatora trajektorii}
	{oprWlasne}
	
Stała czasowa regulatora jest jednak na tyle niska, że w praktyce korpus porusza się bardzo płynnie, bez żadnych zacięć. Podsumowując, cały mechanizm działa dobrze. Testy dowodzą że wszystkie niedogodności związane z pracą danej maszyny są możliwe do wyeliminowania przy pomocy odpowiednich nastaw regulatorów.

